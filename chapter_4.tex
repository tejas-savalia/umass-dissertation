\section{(Partly polished) Introduction}

We naturally categorize items we encounter daily for ease of storage, processing, and decision-making. For example we know instinctively that regardless of shape and form, all lamps form a 'lamp' category based on its function. Several factors determine how we categorize items. Depending on the complexity of rules that determine categories, some categorizations are easier than others \cite{shepard1961learning, nosofsky1994comparing}. Category variability can modulate how often exemplars are classified into that categories \cite{cohen2001category}.

The order of presentation items in category learning tasks has been shown to be an important factor in how category diagnostic features are learned. In particular, when items are presented in a blocked categorical design, participants seem to learn the similarities between the same category items. On the other hand, when items are presented as an interleaved design, participants seem to focus more on learning the features that differentiate the underlying categories \cite{carvalho2017sequence}. As a result of order-dependent differing focus on category diagnostic features, interleaved presentations seem to benefit general category learning. In most prior category-learning tasks assessing order of presentation effects, participants are explicitly asked to learn the underlying categories. There appear to be clear differences when participants focus on learning categories based on how exemplars of these categories are presented \cite{kornell2008learning, kornell2010spacing, whitehead2021transfer, vlach2008spacing, carvalho2014putting, carvalho2017sequence}. In this article, we investigate the effects of order of presentation when category learning is implicit. 

One primary focus on category learning through order of presentation is comparing blocked or interleaved exemplar presentations. For example, \cite{kornell2008learning} showed participants paintings made by two different painters. The order of presentation during exposure was modulated to either be blocked (paintings of one artist shown together followed by the second artist) or interleaved (paintings made by both artists were mixed). When presented with new paintings, and asked which of the two studied artists made them, participants who were exposed to the interleaved format were found to be more accurate at guessing the creator. Category learning also improved for interleaved presentation compared to blocked presentation when tested on items where relevant category features were visually occluded \cite{whitehead2021transfer}. When three-year-old children are tested on the generalization of category-specific features, they appear to benefit from the spaced study of exemplars as compared to a blocked \cite{vlach2008spacing}. By modulating the similarity of presented items, interleaved presentation was found to be better than blocked presentation design on generalization performance particularly when learned exemplars were more visual \cite{kornell2008learning, carvalho2014putting}. 

Interleaved presentation has been theorized to improve in category induction because of context-based variability during encoding \cite{glenberg1979component}. Particularly, for each presented item, an observer will store both the item-specific features along with the context in which the item is encoded. During interleaved presentation, a category diagnostic feature gets encoded under different contexts. Thus, that diagnostic feature will be recalled when tested on novel category items within that context. 

Two theories have been proposed to explain this discriminability-based advantage of interleaving. According to the attention attenuation account, when categories are blocked, participants may think that they have learned the relevant category features after viewing a few items and stop paying attention to additional exemplars of the \cite{kornell2010spacing}. On the other hand, according to the discrimination account, the interleaved presentation allows participants to directly compare the differences between exemplars of different categories that are presented close to each other thereby highlighting these differences \cite{kornell2008learning}. In a direct test \cite{wahlheim2011spacing} found that when participants were shown pairs of exemplars, each belonging to a different category, the interleaving benefit was magnified compared to when they were presented as single items. The authors posit that showing pairs of exemplars would enable participants to carefully study and infer distinctions between category features and hence improve categorization performance .Furthermore, the authors find evidence against the attention attenuation theory by observing that classification performance did not differ as a function of the position in which the exemplar was presented in a stream.

This benefit of interleaved presentation is shown to be modulated by the `level' at which categorization occurs. For example, when \cite{mack2015dynamics} modulated exposure time to individual exemplars along with order of exposure, they found that interleaved presentation was no longer beneficial under short exposure conditions particularly when participants were asked to make a more abstract, `super-ordinate level categorization. On the other hand, when exemplars were presented in a blocked format, a lower, `basic' level categorization was hindered. Thus, category knowledge through order of presentation can be modulated by the level of categorization participants are asked to produce.

It is clear that order of presentation of categories matters during explicit category learning. \textbf{However, the effect of such order of presentation has not been investigated when category learning is implicit}. Indeed recent work shows that participants do acquire category knowledge that when presented implicitly instead of being explicitly asked to learn categories. \cite{unger2022ready} found that assessed on category knowledge, participants appeared to learn category structures without being explicitly instructed to do so. This category knowledge was modulated by the strength of association of the category diagnostic features. \cite{unger2023without} later found that when presented with implicit feature-based categories during a cover task, participants were sensitive towards category diagnostic knowledge. 

More evidence for incidental category learning comes from auditory cognition. \cite{gabay2015incidental} found that participants were sensitive to audio categories learned implicitly as measured by increased reaction times when audio-category-to-response mapping was altered. Incidental category knowledge is further modulated by the sampling category distributions from which exemplars are drawn. \cite{roark2018task} show that probabilistic sampling of exemplars leads to weaker category learning compared to deterministic sampling. Incidental category learning can be further enhanced by task-relevant and disrupted by task-irrelevant feature-to-category mappings \cite{roark2022representational}. Incidental learning may also be disadvantaged compared to supervised intentional learning when categories are non-linearly separable \cite{love2002comparing}.

Thus, while implicit category learning appears to be consistent and dependent on several aspects of the underlying categories, unlike explicit category learning, it is unclear whether implicit category learning enjoys the same advantage when category exemplars are presented in an interleaved vs. a blocked design. Most implicit categorization tasks involve manipulation of features as opposed to manipulation of the temporal order of exposure. In this article, we explore the effects of temporal co-occurrences to manipulate


\paragraph{Other stuff to incorporate:} 

Unsupervised category learning \cite{billman1996unsupervised}, participants could infer rules based on correlating features without being explicitly asked to categorize during exposure. 

Category-related items were recognized better when presented close to each other then when category-unrelated items were \cite{medin1994presentation}. In \cite{medin1987family}, people unsupervised sorted stimuli by a single dimension, ignoring the family resemblance structure. 

SUSTAIN \cite{love2004sustain} seems to explain all these unsupervised category learning phenomena. ALCOVE \cite{kruschke2020alcove} provides for error based diagnostic feature attention learning in GCM \cite{nosofsky2011generalized, nosofsky1986attention}.


\subsection{Participants}


\subsection{Materials}

