We make sense of the world by extracting meaningful information from a continuous sensory stream. Extracting meaningful information involves first segmenting this continuous sensory stream into shorter, processable chunks. These discrete chunks of events represent our recalled experiences and allow us to develop heuristics representing the statistical regularities in our environment.

In this dissertation, I present a predictive context representational account of segmenting the continuous sensory stream into smaller chunks. I demonstrate that maintaining a distributed context representation defined by an expectation of upcoming future events and learned through temporal difference learning naturally leads to the separation of temporally disjoint events without perceptually explicit markers. I contrast this predictive, error-driven account of context representation with an associative learning account and provide behavioral evidence in support of the predictive representational account.

I then show that such predictive context representations can be used as a common framework to understand higher order cognitive processes of event cognition and categorization. I first assess whether implicitly operationalized event boundaries, where changes in ongoing context that mark boundaries are not perceptually salient, provide the same behavioral properties as explicitly operationalized event boundaries thereby providing evidence for shared representations between the two. Finally, I apply the representational framework to understand the cognitive processes behind implicit category learning. I show that predictive representations can arbitrate category learning via the shared temporal context for items in each category. 

Work in this dissertation provides a mechanistic account for statistical learning through widely applicable framework of temporal difference learning. I further demonstrate a use of predictive representations as a common framework to understand higher-order cognitive processes such as event cognition, and categorization. 
