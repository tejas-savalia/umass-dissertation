We experience a stream of continuous information daily. For ease of reference and recall, we break this stream down and store it in meaningful chunks. Event segmentation is a cognitive process that provides mechanisms to break down this temporal stream of information and integrate it across multiple chunks. Prior work has focused on understanding the event cognition process. In the spirit of decomposing cognitive processes into representations and operations on those representations \parencite{cowell2019roadmap}, in this dissertation I explore how this process can be assessed through shared representations of temporal structure in memory and operations on these representations. 

In the second chapter, I investigate how representations of ongoing context can produce behavioral patterns in response times. To that end, I contrast two models of context -- the Successor Representation (SR) as a predictive model that maintains context as an expectation of the future and follows error-driven learning with the Temporal Context Model (TCM) as an associative model that maintains representation as current activity and follows Hebbian learning. I show that the two models produce qualitatively different predictions when exposed to varying amounts of limited information about the environment. I then test these predictions in a serial reaction time task and show that data are more consistent with predictions from the SR model.

\mh{Findings in this chapter provide a theoretical framework with which event boundaries and statistical learning could be studied. Prior work has suggested that slow-down at boundary nodes may be due to a closely related algorithmic process where errors in realization of the temporal structure \parencite{lynn2020abstract}. In addition to the differences in walk lengths explored in this chapter, the predictive SR framework further allows to make testable predictions about information available at different stages of the learning process. SR further provides an implementation level mechanisms via dopaminergic projections to the Hippocampus may allow for error-driven learning of statistical regularities\parencite{stachenfeld2017hippocampus,gershman2018successor}. Future work should aim to directly investigate the differences in predictions between the two algorithmic accounts (erroneous learning vs predictive representation).}

\ac{One major assumption in chapter \ref{chapter-2-walk-lengths-modulate-statistical-learning} warrants some discussion. I assume that higher information contained in a node's representation (derived via SR or TCM) translates to slower reaction times. While there is prior support for slower responses relating higher information in similar tasks \parencite{lynn2020human}, this measure of response time is indirect. It is possible that the reaction time increases at a given node are a result of the source node of the incoming transition rather than the node itself -- previous research have found that as node degree increases, reaction times increase \parencite{lynn2020human}. Nevertheless, in the comparisons tested for experiment 1, source nodes are always non-boundary nodes. Thus, an observed effect must be due to the type of node at which the responses are made. In this work, I made a direct comparison between two models of context representations, the TCM and SR. Future work should also consider other models of context representations and assess how these models of associative memory differ in such statistical learning tasks \parencite{estes1955statistical, mensink1988model, murdock1997context}. Finally, in the experiments used in this chapter, no distinction has been made between what aspect of the participant's experience is associated with a node in the graph in Figure \ref{fig:modular_graph}. Future work should distinguish whether the effects of nodes predicted by models are due to effects on, for example, the visual stimulus associated with those nodes, the motor response associated with that node (which is in turn associated with the visual stimulus), or a combination of both.}

Prior research and findings in event cognition are often limited to tasks where event boundaries are defined by explicit context changes (e.g. change of scene in a movie). Recently, event boundaries have also been shown to be formed without such explicit context changes but through an underlying temporal structure. However, these implicit event boundary tasks are not typically tested using the same tests used for explicit event boundaries. In Chapter \ref{chapter-3-implicit-explicit-event-boundaries}, across two experiments I assess whether implicitly operationalized boundaries share behavioral properties with explicit event boundaries by (1) Testing whether they are remembered better than non-boundaries and (2) Testing whether events across boundaries are perceived farther than those within. Results from the experiment recognition memory experiment provide support for shared representations between implicit and explicit boundaries. Results from the distance judgment experiment do not provide sufficient evidence for such shared representations. However, combined results from both experiments provide support for SR-based context representations for implicit event boundaries.

\mh{Chapter \ref{chapter-3-implicit-explicit-event-boundaries} provides an important indication on shared representations between explicitly operationalized event boundaries and implicitly operationalized event boundaries and whether implicit boundaries are indeed event boundaries. While they have been labeled as boundaries in prior work \parencite*{schapiro2013neural}, work in this chapter provides a more direct test of whether implicit boundaries share behavioral properties of explicit boundaries. The SR modeling framework further provides a representational account for implicit boundaries. Future work should test whether explicit boundaries can be similarly represented through a representational framework such as the SR. While some modeling work has used the associative Context Maintenance and Retrieval Model \parencite{rouhani2020reward} to understand explicit event boundaries, findings in Chapter \ref{chapter-2-walk-lengths-modulate-statistical-learning} suggest that a predictive model may be more appropriate.}

\mh{Chapter \ref{chapter-3-implicit-explicit-event-boundaries} is limited to two tests that are used in explicit boundary paradigms. Future work should aim at testing whether implicit boundaries share other properties of explicit boundaries such as to serve as access points in memory \parencite{michelmann2023evidence}, points in replay \parencite{sols2017event, hahamy2023human}, and points of memory integration \parencite{griffiths2020event}. Systematic assessment of shared properties between implicit and explicit event boundaries will provide further insights into general structure of temporal cognition and pattern recognition.}

Finally in Chapter \ref{chapter-4-category-learning-through-temporal-abstraction}, I show that predictive representation of temporal events (such as SR) can be further used to understand the cognitive process of category learning. Prior findings have suggested that there is a differential benefit to presenting different category exemplars in an interleaved fashion vs blocked fashion in category learning. The SR model provides a representational account for this temporal order of presentation effects by modulating attention towards category diagnostic features. While the precise nature of this modulation operation further depends on the specific visual features that define categories, implicit visual category learning tasks in this chapter show, as predicted by SR, that the temporal order of presentation of category exemplars indeed matters in how categories are learned.

\mh{Work in Chapter \ref{chapter-4-category-learning-through-temporal-abstraction} thus provides a framework for understanding higher order cognition and representation. While categorization and category learning in cognitive psychology is well studied, current work provides a deeper algorithmic insight into how we learn categories in a natural, unsupervised manner. The experimental paradigms used can be further adapted to address other higher order cognitive functions of learning as well. For example, a common problem often addressed through Hierarchical Reinforcement Learning \parencite{botvinick2012hierarchical}, navigation through space can be thought of as achieving a set of sub-goals where all experiences within a sub-goal can be thought of a category.}

\ac{While the framing of studies presented in this dissertation is via event boundaries or statistical learning, the representational framework does not necessarily distinguish between these two processes. Chapters \ref{chapter-2-walk-lengths-modulate-statistical-learning} and \ref{chapter-3-implicit-explicit-event-boundaries} provide an important connection between these two fields of event boundaries and statistical learning (See also \cite{perruchet2006implicit}). Event boundaries, which are implicitly operationalized, are learned over time and provide an important marker to separate statistical regularities in the experienced environment. In turn, extracting of statistical patterns provides a natural break between two statistical regularities that are sufficiently different from each other thereby leading to event boundaries. The representational framework used here provides a way to algorithmically combine the findings in these two fields of cognition.} 

\js{This dissertation assesses shared representations between implicit and explicit event boundaries. One could also argue that boundaries that naturally became explicit were also originally learned implicitly. For a child growing up, the items in the kitchen are no different than the items in the living room. Nevertheless, over time, the child begins to identify the patterns by interacting (or watching others interact) with these items in different contexts. The shared context between kitchen items slowly consolidates into coherent, abstract knowledge that a kitchen is where one eats. Similarly, the shared context within the living room items slowly leads to the abstract knowledge that the living room is where one hangs out (albeit supported by instructions from parents). This way, talking to a friend in the living room becomes a separate event from talking to the same friend in the kitchen. Parent instruction notwithstanding, this formation of boundaries, which are explicit boundaries for adults, was originally acquired by extracting regularities in the environment through statistical learning.}

\js{Statistical learning, a cognitive process through which we acquire patterns in our environment, is widely applicable across multiple domains of cognitive psychology (See \cite{schapiro2015statistical} for a brief review). Our brain supports learning of such regularities automatically and implicitly. Learning the regularities in our environment allows us to develop useful heuristics to make quick decisions when needed. For example, when we visit a new grocery store, we know to generally expect all medications arranged in a specific section separate from a section of food items. This abstract knowledge of how grocery stores work allows us to perform quicker visual searches through aisle headers depending on what we are looking to shop for. The key question this dissertation seeks to answer is what algorithms our brain may implement in order to acquire abstract knowledge of these patterns.}

\ac{Overall, this dissertation shows that representing a temporal sequence of events such that each event represents a prediction of what might happen next, provides \textit{one} reasonable explanation of how we may learn statistical regularities from our environment. This dissertation further shows that such a predictive representation then provides a common framework for investigation into various aspects of human cognition such as forming event boundaries, statistical learning, or categorization. Typically, theoretical work in cognitive psychology is aimed at understanding individual cognitive processes that allow us to function. This representational framework opens a way for cognitive scientists to understand the broader role of such implicit, unsupervised error-driven learning. In future work more cognitive processes such as learning and decision-making in complex environments, the role of cognitive flexibility in behavior, the impact of the environmental experience on emotional and other affective internal states, and others can be distilled into such a common predictive representational account, thereby allowing for a more coherent understanding of human brain function.} 

\section{Broader Implications}



\mh{The spatiotemporal hypothesis in event cognition suggests that items that are closer together in both space and time share representations in the hippocampus \parencite{turk2019hippocampus}. The representational framework proposed in this dissertation suggests that this finding can be extended to other aspects of human psychology as well. For example, the cross-race effect studied in social psychology and eyewitness identification \parencite{young2012perception, wilson2013cross} where recognition for same race faces than different race faces could be studied through a lens of shared representation in how often exposure through development is higher with other faces of the same race.} 

\mh{Errors that lead to learning of of representations such as SR are often linked to dopaminergic signaling in the midbrain \parencite{gershman2018successor}. Future work should test whether increased availability of dopamine during adolescence which has been shown to enhance memory may also allow for a better pattern recognition during that period of development \parencite{cohen2022reward}. Impacts on pattern recognition due to varying dopamine availability over development could have further implications for instruction and in teaching complex concepts across different ages. Similarly, future studies should test whether depleted dopamine levels due to Parkinson's or Huntington's would on the other hand deter pattern recognition and statistical learning.}


\mh{Finally, the focus of studies cognitive psychology (and for this dissertation) has often been limited to participant population in the global north. It is clear that even in meaningless stimuli, the nature of exposure impacts low level cognition of memory and perceptual categorization. Future research must therefore test the generalizability of this representational framework in heterogenous participant population. Recent research in cultural effects on cognition suggests differing processes underlying `lower' level cognition of numbers and time \parencite{pitt2018metaphorical}. Such molding of cognitive processes could also be therefore viewed through the lens of differing representations with consistent operations on those representations as function of different cultural exposures.}

It has been argued that boundaries separating events temporally and spatially share representations. This dissertation further advocates for shared (algorithmic) representations across different processes in cognitive psychology. I show that a common predictive representation framework can be useful in understanding and relating implicit statistical and motor learning (chapter \ref{chapter-2-walk-lengths-modulate-statistical-learning}), event cognition and memory (chapter \ref{chapter-3-implicit-explicit-event-boundaries}), and categorization (chapter \ref{chapter-4-category-learning-through-temporal-abstraction}). Using a common representation can provide an important algorithmic constraint in understanding different cognitive processes. Future work in cognitive psychology can therefore focus on testing operations on these common representations that may account for observed behavior.
