% !TeX root = main.tex

%% This is the "preamble" of the document. This is where the format options get set.
%% Pro-tip: things following the % mark will not be compiled by LaTeX. I'll be using them extensively to explain things as we go.
%% Note: not to scare you off of LaTeX, but it's normal to have problems. And ya girl has been having some. I've included the copyright info at the bottom of the document from the guy who wrote this package, because his documentation doesn't entirely match how it's actually used. So this is a combination of his working preamble along with my added commentary or explanation. 
%%% 
%
%
%%\documentclass[man,12pt,floatsintext]{apa7}
%% Document class input explanation ________________
%% LaTeX files need to start with the document class, so it knows what it's using
%% - This file is using the apa7 document class, as it has a lot of the formatting built in
%% There are two sets of brackets in LaTeX, for each command (the things that start with the slash \ )
%% - The squiggle brackets {} are mandatory for executing the command
%% - The square brackets [] are options for that command. There can be more than one set of square brackets for some commands
%% Options used in this document (general note - for each of these, if you want to use the other options, swap it out in that spot in the square brackets):
%% - stu: this sets the `document mode' as the "student paper" version. Other options are jou (journal), man (manuscript, for journal submission), and doc (a plain document)
%% --- The student setting includes things like 'duedate', 'course', and 'professor' on the title page. If these aren't wanted/needed, use the 'man' setting. It also defaults to including the tables and figures at the end of the document. This can be changed by including the 'floatsintext' option, as I have for you. If the instructor wants those at the end, remove that from the square brackets.
%% --- The manuscript setting is roughly what you would use to submit to a journal, so uses 'date' instead of 'duedate', and doesn't include the 'course' or 'professor' info. As with 'stu', it defaults to putting the tables and figures at the end rather than in text. The same option will bump those images in text.
%% --- Journal ('jou') outputs something similar to a common journal format - double columned text and figurs in place. This can be fun, especially if you are sumbitting this as a writing sample in applications.
%% --- Document ('doc') outputs single columned, single spaced text with figures in place. Another option for producing a more polished looking document as a writing sample.
%% - 12pt: sets the font size to 12pt. Other options are 10pt or 11pt
%% - floatsintext: makes it so tables and figures will appear in text rather than at the end. Unforunately, not having this option set breaks the whole document, and I haven't been able to figure out why. IT's GREAT WHEN THINGS WORK LIKE THEY'RE SUPPOSED TO.
%
%\usepackage[american]{babel}
%
%\usepackage{csquotes} % One of the things you learn about LaTeX is at some level, it's like magic. The references weren't printing as they should without this line, and the guy who wrote the package included it, so here it is. Because LaTeX reasons.
%\usepackage[style=apa,sortcites=true,sorting=nyt,backend=biber]{biblatex}
%\addbibresource{references.bib} % This is the companion file to the main one you're writing. It contains all of the bibliographic info for your references. It's a little bit of a pain to get used to, but once you do, it's the best. Especially if you recycle references between papers. You only have 
%% biblatex: loads the package that will handle the bibliographic info. Other option is natbib, which allows for more customization
%% - style=apa: sets the reference format to use apa (albeit the 6th edition)
%\DeclareLanguageMapping{american}{american-apa} % Gotta make sure we're patriotic up in here. Seriously, though, there can be local variants to how citations are handled, this sets it to the American idiosyncrasies  to get the pieces in the holes once.`
%
%\usepackage[T1]{fontenc} 
%\usepackage{mathptmx} % This is the Times New Roman font, which was the norm back in my day. If you'd like to use a different font, the options are laid out here: https://www.overleaf.com/learn/latex/Font_typefaces
%% Alternately, you can comment out or delete these two commands and just use the Overleaf default font. So many choices!


%% Title page stuff _____________________
%\title{Some Title} % The big, long version of the title for the title page
%\shorttitle{Some Short title} % The short title for the header
%\author{Tejas Savalia}
%\date{\today}
%% \date{January 17, 2024} The student version doesn't use the \date command, for whatever reason
%\affiliation{UMass Amherst}
%%\course{PSY 4321} % LaTeX gets annoyed (i.e., throws a grumble-error) if this is blank, so I put something here. However, if your instructor will mark you off for this being on the title page, you can leave this entry blank (delete the PSY 4321, but leave the command), and just make peace with the error that will happen. It won't break the document.
%\professor{Dr. Professor}  % Same situation as for the course info. Some instructors want this, some absolutely don't and will take off points. So do what you gotta.
%
%\abstract{This is the abstract for this paper, wherein the main points of the introduction, method, results, and discussion are quickly talked about. Probably in more than one sentence, though. Dare I guess, more than two? There is a page break before starting the Introduction.}
%
%%\keywords{APA style, demonstration} % If you need to have keywords for your paper, delete the % at the start of this line
%
%\begin{document}
%\maketitle % This tells LaTeX to make the title page

% \section{Introduction} This command is commented out, because I was taught it was redundant to have the paper's title and introduction together. If your instructor wants it to say "Introduction", delete the % at the start


%% 
%% This is a sample doctoral dissertation.  It shows the appropriate
%% structure for your dissertation.  It should handle most of the
%% strange requirements imposed by the Grad school; like the different
%% handling of titles of one/many appendices.  It will automatically
%% handle the linespacing changes.  The body default is double-spaced
%% (except when you use the singlespace or condensed options).  The
%% default for quotations is single-space, and the default for tabular
%% environments is also single-space.  
%%
%% This class adds the following commands and environments to the
%% report class, upon which it is based:
%% Commands
%% ------------
%% \degree{name}{abbrv} -- Sets the name and abbreviation for the degree.
%%                         These default to ``Doctor of Philosopy''
%%                         and ``Ph.D.'', respectively.
%% \copyrightyear{year} -- for the copyright page.
%% \bachelors{degree}{institution} -- for the abstract
%% \masters{degree}{institution}   --  "
%%     if you have other degrees you may use
%% \secondbachelors{degree}{institution}
%% \thirdbachelors{degree}{institution}
%% \secondmasters{degree}{institution}
%% \thirdmasters{degree}{institution}
%% \priordoctorate{degree}{institution}
%%
%% \committeechair{name}           -- for the signature page
%% or, if you have two co-chairs:
%% \cochairs{first name}{second name}
%%
%% \firstreader{name}              --  "
%% \secondreader{name}             --  "
%% \thirdreader{name}              -- (optional)
%% \fourthreader{name}             --  "
%% \fifthreader{name}              --  "
%% \sixthreader{name}              --  "
%% \departmentchair{name}          -- for the signature page
%% \departmentname{name}           --  "
%%
%% \copyrightpage                  -- produces the copyright page
%% \signaturepage                  -- produces the signature page
%%
%% \frontmatter                    -- these are required in their various
%% \mainmatter                     -- appropriate locations
%% \backmatter                     --
%%
%% \unnumberedchapter[toc]{name}   -- like \chapter, except that it
%%                                    produces an unnumbered chapter;
%%                                    alternatively, like \chapter*,
%%                                    except that it lists the chapter
%%                                    in the table of contents.
%%
%% New environments:
%%   dedication  -- for the dedication
%%   abstract    -- for the abstract
%%
%% The thesis documentclass is built on top of the report document class.
%% It accepts all of the options that the report class accepts, plus the
%% following:
%%     doublespace -- the default, indicates double spacing as per U.Mass.
%%                    requirements.  You will need this when you do your
%%                    final copy.
%%     singlespace -- for earlier work, not acceptable to the Grad school
%%     condensed   -- for earlier work, not acceptable to the Grad school,
%%                    creates condensed versions of the frontmatter. 
%%                    Condensed implies singlespace.
%%     dissertation - the default, indicates that this document is a
%%                    dissertation.
%%     proposal    -- indicates that this document is a dissertation proposal,
%%                    rather than a dissertation.  This will only change the
%%                    wording on the title and signature pages.
%%     thesis      -- indicates that this document is a Master's thesis 
%%                    rather than a doctoral dissertation.  This also changes
%%                    the default for \degree to Master of Science, M.S.
%%     allowlisthypenation -- (the default), allows hyphenation of words in
%%                    the table of contents, the list of figures, and the list
%%                    of tables.  I believe that this is acceptable to the 
%%                    Graduate School.
%%     nolisthyphenation -- disallows hyphenation of words in the table of
%%                    contents and the list of figures and tables.  Use this 
%%                    option if the Grad School doesn't like your hyphenation.
%%     nicerdraft  -- relaxes some of the Grad School's rules for working with
%%                    drafts -- has no effect when doublespace is in effect
%%     nonicerdraft -- the default, leaves things in draft as they will be in
%%                     the final version
%% umassthesis changes the default font size to 12pt, but you may specify 10pt or
%%   11pt in the options.
\documentclass[dissertation]{umassthesis}          % for Ph.D. dissertation or proposal
% \documentclass[thesis]{umassthesis}  % for Master's thesis

%%
%% If you have enough figures or tables that you run out of space for their
%% numbers in the List of Tables or List of figures, you can use the following
%% command to adjust the space left for numbers.  The default is shown:
%%
%% \setlength{\tablenumberwidth}{2.3em}

%% Use the hyperref package if you're producing a version for online
%% distribution and you want hyperlinks.  Note that the Grad School doesn't want
%% their PDF viewers to colorize or otherwise highlight the links; use the
%% hidelinks option to hyperref to avoid decorating links.
%\usepackage[hidelinks]{hyperref}

%% One way of formatting the epigraph/frontispiece is to use this package.
%\usepackage{epigraph}
% \usepackage{biblatex}
\usepackage{xcolor}
\usepackage{float}
\usepackage{amsmath}
\usepackage[hidelinks]{hyperref}
\usepackage{graphicx}
\usepackage{booktabs}
\usepackage{multirow}
\usepackage{soul}

\usepackage[style=apa, backend=biber]{biblatex}
\addbibresource{references.bib}

\newcommand{\ac}[1]{\textcolor{black}{#1}}
\newcommand{\js}[1]{\textcolor{black}{#1}}
\newcommand{\yk}[1]{\textcolor{black}{#1}}
\newcommand{\mh}[1]{\textcolor{black}{#1}}

% \usepackage{tikz}
% \usepackage{subcaption}


% \bibliographystyle{umassthesis}
% \usepackage{apa7}

\begin{document}
	
%%
%% You must fill in all of these appropriately
\title{Investigating the Role of Predictive Representations in Implicit Event Boundaries, Statistical Learning, and Categorization.}
\author{Tejas Savalia}
\date{September 2024} % The date you'll actually graduate -- must be
% February, May, or September
\copyrightyear{2024}
\bachelors{B.E.}{Gujarat Technological University}
\masters{M.S.}{International Institute of Information Technology, Hyderabad}
\secondmasters{M.S.}{University of Massachusetts Amherst}
%\priordoctorate{M.D.}{University of Never-never-land}
\committeechair{Andrew Cohen}
%\cochairs{B. B. Bahh}{I. M. A. Wolf}
\firstreader{Jeffrey Starns}
\secondreader{Youngbin Kwak}
\thirdreader{Meghan Huber}
%\fourthreader{}   % Optional
%\fifthreader{}            % Optional
%\sixthreader{}            % Optional
\departmentchair[Chair]{Maureen Perry-Jenkins} % Default uses "Department Chair" as the title. To
% use an alternate title, such as "Chair", use \departmentchair[Chair]{Pete Shearer}
% CICS uses "Chair of the Faculty" as of 2019.
\departmentname{Psychological and Brain Sciences}
% \departmentname{Robert and Donna Manning College of \\Information and Computer Sciences}


%% If your degree is something other than a Ph.D. (for a dissertation), or
%% an M.S. (for a thesis), you will need to uncomment the appropriate
%% following line:
%%
%% \degree{Doctor of Education}{Ed.D.}
%% \degree{Doctor of Philosophy}{Ph.D.}
%%
%% \degree{Master of Arts}{M.A.}
%% \degree{Master of Arts in Teaching}{M.A.T.}
%% \degree{Master of Business Administration}{M.B.A.}
%% \degree{Master of Education}{M.Ed.}
%% \degree{Master of Fine Arts}{M.F.A.}
%% \degree{Master of Landscape Architecture}{M.L.A.}
%% \degree{Master of Music}{M.M.}
%% \degree{Master of Public Administration}{M.P.A.}
%%\degree{Master of Public Health}{M.P.H.}
%% \degree{Master of Regional Planning}{M.R.P.}
%% \degree{Master of Science}{M.S.}
%% \degree{Master of Science in Accounting}{M.S. Acctg.}
%% \degree{Master of Science in Chemical Engineering}{M.S. Ch.E.}
%% \degree{Master of Science in Civil Engineering}{M.S.C.E.}
%% \degree{Master of Science in Electrical and Computer Engineering}{M.S.E.C.E.}
%% \degree{Master of Science in Engineering Management}{M.S. Eng. Mgt.}
%% \degree{Master of Science in Environmental Engineering}{M.S. Env. E.}
%% \degree{Master of Science in Industrial Engineering and Operations Research}{M.S.I.E.O.R.}
%% \degree{Master of Science in Manufacturing Engineering}{M.S. Mfg. Eng.}
%% \degree{Master of Science in Mechanical Engineering}{M.S.M.E.}
%%
%% \degree{Professional Master of Business Administration}{P.M.B.A.}


%%
%% These lines produce the title, copyright, and signature pages.
%% They are Mandatory; except that you could leave out the copyright page
%% if you were preparing an M.S. thesis instead of a PhD dissertation.
\frontmatter
\maketitle
\copyrightpage     %% not required for an M.S. thesis
\signaturepage

%%
%% Dedication is optional -- but this is how you create it
% \begin{dedication}              % Dedication page
% \begin{center}
% 	\emph{To those little lost sheep.}
% \end{center}
% \end{dedication}

%%
%% Epigraph (aka frontispiece) is also optional, but this is one way you
%% can create it
%\begin{frontispiece}
%  %% Format to your liking -- see documentation of epigraph package
%  \setlength{\epigraphrule}{0pt}
%
%  \begin{epigraphs}
%    \qitem{%
	%      \itshape
	%      Mary had a little lamb,\\
	%      Her fleece was white as snow.\\
	%      \vspace{\baselineskip}
	%      And everywhere that Mary went\\
	%      The lamb was sure to go.
	%      \vspace{\baselineskip}}
%    {Sarah Josepha Hale}
%
%    \vspace{2\baselineskip}
%    \qitem{%
	%      \itshape
	%      Baa, baa, black sheep,\\
	%      Have you any wool?\\
	%      Yes, sir, yes, sir,\\
	%      Three bags full;\\
	%      One for the master,\\
	%      And one for the dame,\\
	%      And one for the little boy\\
	%      Who lives down the lane.
	%      \vspace{\baselineskip}}
%    {English Nursery Rhyme}
%
%  \end{epigraphs}
%\end{frontispiece}

%%
%% Acknowledgements are optional...yeah, right.
\chapter{Acknowledgments}             % Acknowledgements page
Work done during my PhD would not have been possible without the contribution of many people. This dissertation is indeed a baby which required a village to take care of.

I would first like to thank my advisors who saw this dissertation through. None of the work presented here would have been possible without Andrew Cohen and Jeffrey Starns who over countless meetings have helped me hone my projects, and ensured scientific rigor. They also let me work on whatever I wanted, provided the resources I needed, and generally made research fun. I would also like to thank Youngbin Kwak and Meghan Huber who served on my dissertation committee -- first for continuing on it as I changed projects and for providing invaluable feedback that significantly improved my dissertation. My first advisors in grad school at UMass; David Huber and Rosie Cowell taught me the fundamental steps of research in my early years as a PhD student really honed my ideas and allowed me to grow as a researcher. Other cognitive faculty at UMass Amherst provided invaluable experience in academic presentation through brown-bag questions and through courses I took over the years. 

I would next like to thank my \st{collegues} friends in the cognitive program throughout the years: Sean, Anna, Melisa, Mar, John, Jerome, Michael, Yun, Chiungyu, Kuan-Jung, and Sandarsh for being a stable presence in this intellectual journey. You made this program feel like home. I would also like to call out Natasha and Kuan Jung, my co-hort mates who went through the roller coaster with me. 

Weekly board games in PBS is one of the most consistent social activity I developed is at UMass and the group of friends have practically served as my family away from home. Anna, Clara, Tori, Sean, Trina, Fran, Ramiro, and Kuan Jung; your dedication to put up with my, ahem, eccentric, hobby has never failed to lift my spirits and always provided with an outlet I look forward to every week. The last two years of my grad school would not have been as much fun if not for you and I am glad that this tradition is set to continue going forward. Finally, I'd be remiss if I did not mention the `tiny living room' folks; Ramiro, Sandarsh, and Hyejoo for being my first non-work friends at work. 

Several people helped me through grad school before and through the pandemic -- Aarohi, Meet, Rik, Sohini, Pracheta, and Princy, I will never forget the absolute random conversations, and trips that provided a venue to turn my mind off work and let me stay sane. Kunal, while you were not at UMass, you have never let go of this friendship with regular check ins, and truly showed me the value of real friendship. I don't know how you do it, but please continue doing it.

Manasa -- I can write another dissertation to describe your tireless support. Instead, I will just say that absolutely no part of the dissertation would have been possible without you. You helped me in literally everything I wanted your help. From non-work ups and downs, to serving as an initial ideas sounding board, to helping me with analyses and programming, to feedback on my writing, and through celebrating me, you have seen it all, done it all and been my consistent cheerleader in these six years. I could not have chosen a better human being to do so. Thank you.

Finally, I would not have been here if not for continued, unwavering support from my family. Mummy, pappa, diku, Nishita, and Aaru -- thank you for letting me do something regardless of whether it is clear what it is I exactly do. Your faith in me has been the key to make this work happen.



%%
%% Abstract is MANDATORY. -- Except for MS theses
\begin{abstract}                % Abstract
We make sense of the world by extracting meaningful information from a continuous sensory stream. Extracting meaningful information involves first segmenting this continuous sensory stream into shorter, processable chunks. These discrete chunks of events represent our recalled experiences and allow us to develop heuristics representing the statistical regularities in our environment.

In this dissertation, I present a predictive context representational account of segmenting the continuous sensory stream into smaller chunks. I demonstrate that maintaining a distributed context representation defined by an expectation of upcoming future events and learned through temporal difference learning naturally leads to the separation of temporally disjoint events without perceptually explicit markers. I contrast this predictive, error-driven account of context representation with an associative learning account and provide behavioral evidence in support of the predictive representational account.

I then show that such predictive context representations can be used as a common framework to understand higher order cognitive processes of event cognition and categorization. I first assess whether implicitly operationalized event boundaries, where changes in ongoing context that mark boundaries are not perceptually salient, provide the same behavioral properties as explicitly operationalized event boundaries thereby providing evidence for shared representations between the two. Finally, I apply the representational framework to understand the cognitive processes behind implicit category learning. I show that predictive representations can arbitrate category learning via the shared temporal context for items in each category. 

Work in this dissertation provides a mechanistic account for statistical learning through widely applicable framework of temporal difference learning. I further demonstrate a use of predictive representations as a common framework to understand higher-order cognitive processes such as event cognition, and categorization. 

\end{abstract}

%%
%% Preface goes here...would be just like Acknowledgements -- optional
%% \chapter{Preface} 
%% ...


%%
%% Table of contents is mandatory, lists of tables and figures are 
%% mandatory if you have any tables or figures; must be in this order.
\tableofcontents                % Table of contents
\listoftables                   % List of Tables
\listoffigures                  % List of Figures

%%
%% We don't handle List of Abbreviations
%% We don't handle Glossary

%%%%%%%%%%%%%%%%%%%%%%%%%%%%%%%%%%%%%%%%%%%%%%%%%%%%%%%%%%%%%%%%%%%%%%%%%
%% Time for the body of the dissertation
\mainmatter   %% <-- This line is mandatory


\chapter{Introduction}
\label{introduction}
We experience a constant stream of sensory information from the moment we are born. Our brain parses this information and slowly learns to extract meaning from it. From recognizing the mother's scent as a survival instinct to formulating complex plans to defeat a board game opponent, our brain extracts meaning from our surroundings, considers prior experience and the current state of the world, and makes decisions to interact accordingly. My key question in this dissertation is: How do we learn to extract meaning from our surroundings?

Extracting meaningful information from our experience is beneficial to our functioning and survival. When approached by a cheetah in a forest, we do not wait to evaluate the exact number of spots on its skin before deciding to run. Such abstractions and formation of heuristics allow us to process naturally complex surroundings in quick time and act accordingly.

When extracting meaning from our surroundings, we often (need to) ignore minute details and abstract out towards a coherent thought of our surroundings. The extreme example above aside, we observe such abstractions in almost all day-to-day activities. Imagine someone asking you about the events during the day before reading this manuscript. Perhaps you are reading it on your office computer and you recall a sequence of events starting from waking up, to getting ready, driving or taking public transit to work, getting your morning coffee and breakfast in with an email check before turning your attention to this manuscript. Each event described above combines several sub-events that are abstracted away in such a verbal recall and description. For example, getting ready involves several steps from brushing your teeth, showering, and wearing your work outfits. Each sub-event can be further thought of as an abstraction from sub-sub-events -- brushing your teeth is a combination of putting toothpaste on the brush head, the physical act of brushing, followed by rinsing. While we perform each act continuously in time, our recall (and by extension, representation in memory) of these past events is discrete, segmented, and abstracted. The meaningfully separable chunks of a continuous stream of events help in storage in (and retrieval from) memory.

There is often agreement on what it means for a chunk and for boundaries defining such chunks to be `meaningful'. In the example above, it is reasonable to argue that brushing teeth, showering, and putting on clothes are three distinct activities. Furthermore, even when the true transitions between these activities are continuous and seamless to a  independent, naive observer, the boundaries between these events are perceptually meaningful. What aspect of the environment dictates this agreement about the points at which we segment events and what is special about the properties of the events between those points that lead to distinct representations in memory?

One could argue that these events that occur at different points in time also occur at different spatial locations, thus providing different contexts and hence separate representations in our brain to be considered distinct. Transitioning from one (temporal) event to another can thus be akin to transitioning from one room of the house to another. However, it is almost impossible to decouple temporal events from spatial events assuming a causality from spatial segmentation to temporal segmentation for any spatially experienced distinct event is also a temporally experienced distinct event. Instead, arguing that temporally distinct events lead to a spatially distinct representation can provide a more encompassing explanation of distinct representations for events in distinct spatial \emph{and} temporal contexts. One could similarly argue that the formation of ``meaningful'' chunks is through perceptual differences between the events we experience. While lower-level perceptual experiences are indeed often different for different events, the mapping of perceptual differences onto segmented events is arbitrary and not a \emph{sufficient} condition. For example, eating an apple is recalled as eating an apple regardless of whether it has a green leaf added to its top. Perceptual distinctions are not enough to determine whether events are represented distinctly. What then is the key mechanism that leads to events being segmented? 

In this dissertation, I argue that the primary reason and mechanism through which we segment events is based on temporal contingencies of various sub-events that encompass an event. Specifically, we recall being in the kitchen as different from being in the bedroom because we have a coherent set of experiences in the kitchen that are distinct from a coherent set of experiences in the bedroom. For an infant forming knowledge of the world, a kitchen while perceptually distinct from a bedroom, is not meaningfully different. With experience and observation of the functions within these spatially (and perceptually) distinct locations, the child slowly develops distinct representations of the two rooms.

This dissertation focuses on understanding temporal contingencies' role in event segmentation, and by extension, general pattern extraction. I argue that even \emph{without} any spatial or perceptual information that may aid us in separating events in memory, we can use temporal coherence to experience separate events and abstract information to aid higher-level cognition. Specifically, I investigate the parsing of a continuous stream of information into discrete chunks in three ways:



\begin{itemize}
	\item
	The possible algorithmic representations that naturally lead to such segmentation and the impact of environmental properties in aiding this abstraction.
	\item
	The properties of the temporal boundaries when such temporal abstraction occurs naturally and implicitly.
	\item
	The role of temporal events separated by underlying transition structures in forming higher-order abstractions such as categories.
\end{itemize}

\section{Scope of this dissertation}\label{scope-of-this-dissertation}

The human brain is a complex machine -- millions of neurons act as computational units and combine in specific ways to form a functioning human being. These neurons come together to implement several levels of function from lower-level automatic perception to higher-level planning and conscious thought. This dissertation does \textbf{not} focus on these implementational-level mechanisms of cognition. Rather, it focuses on \emph{algorithmic} computations that neurons may, collectively, implement that lead to us acquiring patterns in the environment around us \parencite{marr1976understanding}. \mh{These algorithmic computations are then experimentally tested through behavioral data and computational modeling.}

Analyses in this dissertation use several models of cognition in investigating the role of implicit statistical learning. In most cases, the focus of this dissertation is \textbf{not} to evaluate the validity of these models. Indeed, most models of cognition are wrong but are useful \parencite{fisher2019all} and I use several such models to evaluate specific aspects of how we acquire patterns. Similarly, this dissertation proposes modifications to the previously known models based on context representations. These modifications are solely to derive predictions and provide possible explanations of findings from these models and are not rigorously tested. Future work should aim to test the updates proposed in this work and therefore provide more holistic explanation of the cognitive processes explored in this manuscript. \ac{Nevertheless, models and proposed modifications to these models used in this dissertation play a key role in generating experimentally testable predictions presented in this dissertation which are then used to motivate experimental designs. }

Finally, the data collected and used in this dissertation is much more rich than presented. In order to limit the scope to the specific questions of interest, only a subset of analyses involving simpler (often linear) models is presented. Future work will incorporate more complex models on this data to extract fine grained information of the representation driven cognitive processes.

\mh{The key contribution of this dissertation is in presenting a representational framework to understand the cognitive processes of pattern recognition and statistical learning. Furthermore, this dissertation provides for a common representational framework, with prior evidence for this representational framework being implemented in the brain \parencite{gershman2018successor} to understand a myriad of cognitive processes at different levels of cognition. In most prior work, cognitive processes of event cognition, learning, memory, and categorization in prior work have largely been studied separately within those sub-fields. However, this dissertation shows that one common representational framework of learning can be used to explain these processes by varying operations on that representation \parencite{cowell2019roadmap}.
}

\section{Format of this dissertation}\label{format-of-this-dissertation}

In the rest of this dissertation, I present three lines of studies investigating the role of implicit temporal boundaries in cognition. In Chapter 2, I present an algorithmic representational framework that naturally leads to a representation of separable events without the need to rely on explicit properties of the experienced events. I also show that this predictive framework allows for a distinct representation of event boundaries as special events and contrast it with an associative representation. In Chapter 3, I present work comparing the properties of these event boundaries which are operationalized implicitly (i.e. through no perceptually special information) with event boundaries as they have been studied in prior literature which are operationalized explicitly. In Chapter 4, I present work investigating the role of these implicitly operationalized event boundaries in categorization to serve as a gateway for understanding higher-order cognition in the context of temporal segmentation and pattern acquisition. \mh{Finally, in Chapter 5, I present a summary of findings presented in this dissertation and provide an overview of broader implications of this work.}


\newpage

\chapter{Quality of Environmental Exposure modulates statistical learning}
\label{chapter-2-walk-lengths-modulate-statistical-learning}
\section{Introduction}

Imagine you just moved to the United States and are visiting Target for the first time. Perhaps since you just moved in, your first goal is to furnish your apartment. You look around at the entrance, and navigate your way to the furniture section perhaps while taking a few false turns on the way, buy the stuff you need, pay and leave. The next time you visit for, for example, groceries and produce. You visit again for sporting goods and then again for gifts for your friends. A year later, all such subset of visits through the Target store makes you an expert in knowing the specific route to the section you need to visit. What algorithmic mechanisms allow us to build such expertise to create connections in an explored environment even when during no single visit, you explored all possible connections between regions of the store?

We are able to build up a map of the environment without being exposed to the full extent of it in a single iteration simply based on local exposures to it. Such building is often implicit -- you just know that in order to go to the furniture area, you need to pass through the gifts even when you did not explicitly explore this connection before. \mh{In this chapter I explore two algorithmic mechanisms that can be used for abtracting structural information from local exposures. I compare the Successor Representation \parencite{dayan1993improving}, and the Temporal Context Model \parencite{howard2005temporal} to generate quantifiable predictions from predictive and associative representations of context, respectively. I then experimentally test these predictions and show that humans rely on predictive representations to learn the global environmental structure from local exposure.}

The effect of local exposure in acquiring structural knowledge of the environment have been explored in several areas of cognitive psychology through artificial grammar and language learning \parencite{knowlton1992intact, romberg2010statistical, aslin2012statistical, dehaene2015neural}, visual statistical learning \parencite{fiser2002statistical, turk2008multidimensional, brady2008statistical}, or motor sequence learning \parencite{baldwin2008segmenting, nissen1987attentional, cleeremans1991learning, kahn2018network}. In recent work, (implicit) acquisition of higher order knowledge of the environment from lower order exposure is studied through structured graph based transitions between stimuli \parencite{schapiro2013neural, karuza2017process, kahn2018network, lynn2020humans, lynn2020human, karuza2022value}. For example, \cite{schapiro2013neural} \ac{\js{asked participants to study a stream of stimuli based on a connected modular graph \ac{\textbf{such as the one in}} in Figure \ref{fig:modular_graph}. Each stimulus \ac{\textbf{in their experiment}} was associated with a node of the graph (blue circles) and the stream was generated through a random walk where each subsequent stimulus was randomly chosen from the connected neighbors of the current node. Participants were then asked to hit a key wherever it felt like a `natural break'. Participants often parsed the edges that connect two modules as `natural breaks' even when their local exposure does not distinguish between the cross-module \ac{\textbf{e.g. edge between nodes 0 and 14 in Figure \ref{fig:modular_graph}}} and within-module edges \ac{\textbf{e.g. edge between nodes 7 and 8 in Figure \ref{fig:modular_graph}}}.}} 

More commonly, global-scale structure acquisition \textbf{\ac{(i.e. when participants are said to have acquired (implicit or explicit) knowledge of the graph in Figure \ref{fig:modular_graph})}} has been \textbf{\ac{measured through response times}}. Earlier work in serial reaction time tasks shows that breaking an implicitly learned motor sequence leads to slower reaction times \parencite{nissen1987attentional, cleeremans1991learning}. The slowed reaction times when crossing the between-module edges have also been shown in recent work on statistical learning in modular graph structures \parencite{kahn2018network, lynn2020humans, karuza2017process, karuza2022value, karuza2019human, lynn2020human}. 

This slowdown across module edges appears to be mediated by the nature of the walk experienced across the community structure where random and Eulerian walk (a walk where each edge of the graph is visited exactly once before repeats) experiences continue to show this slowdown whereas a Hamiltonian walk (a walk were each node of the graph is visited exactly once before repeats) experience does not \parencite{karuza2017process}. Thus it appears that the kind of experience through the graph alters the knowledge of underlying statistical patterns. Similarly, the topographical structure of a graph in motor skill learning tasks also appears to alter structural knowledge \parencite{lynn2020abstract, lynn2020human, lynn2020humans} where modular graphs like in Figure \ref{fig:modular_graph} produce the largest dip in reaction times when responding to boundary items. 

\mh{While these effects are not unique to the modular graph in Figure \ref{fig:modular_graph} and graphs of various topological variations produce similar effects \parencite{karuza2019human}, for the remainder of this work, we focus on using the same modular graph used in \cite{schapiro2013neural}. The graph not only provides both the desired modular structure that can translate to pattern extraction, it also provides useful symmetry in the number of nodes in each cluster, the degree of the graph \textbf{\ac{(equal number of connections at each node), and in allowing for tractability of stimuli associated with each node given constraints on working memory and learning capacity}}.}

\begin{figure}[ht]
	\centering
	\caption{Modular graph structure used in \cite{schapiro2013neural}. Locally, each node is connected to four nodes with each edge equally probable. However, globally, the graph structure consists of three sub-modules interconnected through `boundary nodes'}
	\includegraphics[width = \textwidth]{chapter_notebooks/chapter_2/figures/modular_graph.png}
	\label{fig:modular_graph}
\end{figure}

Why do we slow down at boundary nodes that lead to the adjacent module even when the local probability of that particular transition is the same as any other transitions? Understanding this particular property of human behavior may provide deeper insights into the kind of representations that lead to global-scale structure acquisition. Stimuli in tasks typically used in such statistical learning paradigms are either not meaningless or randomly assigned to each node -- the only difference between the boundary node and other non-boundary nodes is in context of the global structure of the graph. The event boundary literature (where boundaries are typically operationalized through explicit changes in context) suggests that boundaries alter the predictability of future events and this predictability leads to event segmentation \parencite{zacks2007event, clewett2019transcending}. Thus, in implicitly operationalized boundaries such as in serial reaction time tasks, the slowdown at the boundary node may imply a similarly increased uncertainty at boundary nodes leading to slowed responses. Prior work aimed at understanding human representation of graph structures indeed points to an increased `cross-entropy' between a learner's estimate of the transition probability and the true transition probability of the environment \parencite{lynn2020abstract, lynn2020humans, lynn2020human}. 

%In addition to being uncertain about the immediate next stimulus, participants are also uncertain about switching to a neighboring cluster or staying within the same cluster.

In particular, \cite{lynn2020human} show that algorithms of contextual representations such as the Successor Representation (SR) model in Reinforcement Learning \parencite{dayan1993improving, momennejad2017successor, gershman2018successor} or the associative learning based Temporal Context Model (TCM) can naturally lead to an increased cross-entropy for cross-cluster transitions relative to within-cluster transitions in modular graphs. In the current work, by using the framework of entropy as a proxy for an estimate reaction times in a modular graph we aim to 1) Experimentally test the predictions of these two models when exposure through the modular graph structure of is partial and 2) Identify which of the two models of representation best explain the observed data. 

\subsection{Representations of Temporal Context}

\subsubsection*{Successor Representation}
\label{successor-representation}

The Successor Representation (SR) model of reinforcement learning has been used as a model to understand the generalization of reinforcement learning behavior in large action spaces \parencite{dayan1993improving}. In recent work, the SR model has also been shown to be a reliable model for explaining human decision-making behavior in multi-step environments. The model's \yk{mechanism of model-free, trial and error learning of transition probabilities and model-based learning of rewards} accurately predicts that humans are worse at adapting to changes in the transition probability of a learned environment than to changes in the end-point rewards \parencite{momennejad2017successor}. There has been further evidence of SR being represented in the Hippocampal cells which represent space \parencite{gershman2018successor, stachenfeld2017hippocampus}.

Briefly, the SR model represents each state in the actionable space as a vector of predictive representations. For an environment of $N$ discrete states, the SR matrix $M$ of size $(N X N)$ maintains expected future visits to a given state from each state. Specifically, element $M_{i,j}$ of the matrix represents the expected future visits to state $j$ from state $i$. This transition matrix is learned over time based on the temporal difference error learning rule \parencite{sutton2018reinforcement}. For example, consider at a given point in time, $t$, an agent maintaining the SR matrix is in state $i$. The agent now moves to state $j$ out of the possible $N$ states. The $i^{th}$ row of the SR matrix is updated as follows:

\begin{equation}
	\hat{M}_{i,j} \leftarrow \hat{M}_{i,j} + \alpha[\delta(s_{t+1},j) + \gamma*\hat{M}_{s_{t+1},j} - \hat{M}_{s_t,j}]
\end{equation}

where $\delta(., .)$ equates to 1 if both arguments are equal otherwise it equates to 0. Thus, the matrix increases the probability of visiting a state $j$ from state $i$ if state $j$ is visited in the current experience and it decreases the probability of visiting all other states from state $i$. Parameter $\alpha$ is a learning rate parameter that determines how much of the previous estimate of visiting state $j$ from $i$ is factored into the current update. Parameter $\gamma$ is a future discount parameter that dictates how much in the future the agent sees -- specifically, a higher value of $\gamma$ indicates future visitations to state $j$ are weighed high in the current update.

\ac{For example, let's assume that the entire world (from the perspective of a participant) constitutes the 15 items they will see in the study. Now, as they start their experience of a random walk on the graph in Figure \ref{fig:modular_graph}, they will initially have no information about the transition probability structure. Thus, they will assign an equal probability across all possible transitions (i.e., $\frac{1}{15}$ between each pair of nodes/stimuli). Let's say they experienced a transition from node 1 to node 2. The prediction error for observed transition will be positive (example below)}:
\begin{equation}
	\begin{aligned}
		\Delta\hat{M}_{1,2} = 1 + \gamma*\hat{M}_{s_{2, 2}} - \hat{M}_{1,2} \\
		 = 1 + \gamma*\frac{1}{15} - \frac{1}{15} > 0
	\end{aligned}
\end{equation}

\ac{Similarly the prediction error for all transitions that were \textit{not} observed from node 1 will be negative (example below)}:

\begin{equation}
	\Delta\hat{M}_{1,3} = 0 + \gamma*\hat{M}_{s_{2, 3}} - \hat{M}_{1,3} \\
		= \gamma*\frac{1}{15} - \frac{1}{15} < 0
\end{equation}

\ac{Note that in the above equations, the model also accounts for a future transitions $M(2, 2), \& M(2, 3)$ while updating its expectation of transition $M(1, 2)$, and $M(1, 3)$ respectively. This way, any current transition impacts the expected trainstitions to the future nodes as well. \textbf{By allowing for transitions expected in the future to weigh in on current updates, this learned matrix allows for inferring for transitive properties in the graph structure.}}


\subsubsection*{Temporal Context Model}
The Temporal Context Model (TCM) was devised to explain the primacy and recency effects in human recall and recognition memory \parencite{howard2005temporal}. \ac{The TCM model assumes that the items or stimuli shown to a participant during a study phase through a sequential exposure maintain a temporal context \textbf{as a vector of activity of all stimuli in the experiment}. As new items get encoded, existing context from the previous items allows the new items to be bound to the previously seen items thereby sharing the temporal context}. Briefly, the TCM can be formalized as in \cite{gershman2012successor}:

\begin{equation}
	\begin{aligned}
		t_n = \rho * t_{n-1} + f_n \\ 
		\hat{M}_{i, j} \leftarrow \hat{M}_{i, j} + \alpha f_{n+1} t_{n, i}			
	\end{aligned}
\end{equation}

where $t_n$ is said to be a `context' vector for item $n$. The context drift parameter $\rho$ determines the proportion of the previous elements's context that gets incorporated in the current context. $f_n$ is a one-hot encoded vector for item $n$ \js{-- element of the $N$ item-vector $f_n$ is 1 for the item it represents and 0 otherwise}. The learning rate parameter $\alpha$ determines what proportion of the currently experienced state binds with the existing context. 

\ac{For a similar experience as described in the example for the SR model, the the TCM context vector $t_n$ is first updated based on the prior, existing context $t_{n-1}$. For the experienced transition from node 1 to 2 and assuming the current transition to node 1 came from node 0:}

\begin{equation}
	\begin{aligned}
		t_1 = \rho*t_{n-1} + 1  \\
		\hat{M}_{1, 2} = \hat{M}_{1, 2} + \alpha*1*t_{1,1} \\ 
	\end{aligned}
\end{equation}

\ac{Note that for a transition that is \textit{not} experienced, the cell $\hat{M}_{i, j}$ does not get updated as $f_{n+1}$ would be $0$.}

The key difference between the two models of temporal context is two fold: (1) SR Relies on error-based learning whereas TCM relies on hebbian, associative learning and (2) Through the future discount parameter $\gamma$, SR also learns the predictability observing states in the near future based on the locally experienced transitions. This future discount parameter in SR thus allows the model to represent transitive associations as well  \parencite{gershman2012successor}. 

\subsection{Model Simulations}

\mh{The differences between the models stated above lead to differing representations of learned temporal structure when the models are exposed through the random walk in Figure \ref{fig:modular_graph}. \ac{\textbf{To preview, representations associated with boundary nodes carry more information in the SR than the TCM.}}}

\mh{Simulating the models described above can thus lead to an estimate of expected behavior from both these models. Differences in these expected behaviors thus allow us to generate experimentally testable hypotheses.} Figure \ref{fig:SR-TCM-model-simulations} shows the context matrix representation after the models have been simulated for a random walk through the graph structure in Figure \ref{fig:modular_graph} as a result of a random walk after 1000 trials for both models. \ac{The matrices of Figure \ref{fig:SR-TCM-model-simulations} represent the model predictions of context representations. Each row corresponds to a context representation of that node (of the 15 total nodes). The `activity' levels in each cell indicated by the heatmap represent the relative proportion of nodes that are active when a particular node is visited.} 

\ac{\js{Parameters used in the simulations shown in Figure \ref{fig:SR-TCM-model-simulations} were determined by a Representational Similarity Analyses style procedure \parencite{kriegeskorte2008representational}. Specifically, for a combination of parameters over a valid range, a distance matrix was computed \textbf{\ac{to represent}} an euclidean distance between each pairs of rows of the generated context matrix (SR or TCM). Parameters that maximize the correlation between this generated distance matrix and the true distance matrix (derived from transition matrix representing equal probability transitions across the connected edges in the modular graph of Figure \ref{fig:modular_graph}) were chosen via a grid search.}} 

\begin{figure}[!ht]
	\centering
	\includegraphics[width = 0.9\textwidth]{chapter_notebooks/chapter_2/figures/SR_vs_TCM_Matrices.png}
	\caption{Successor Representation and Temporal Context Model representations of context following a random walk through the modular graph structure.}
	\label{fig:SR-TCM-model-simulations}
\end{figure}


\ac{By using a behavioral measure of response times, previous work has shown that participants can acquire the global structure of the graph for a random walk. Particularly, as participants acquire knowledge of the underlying structure, their reaction times are slower in responses following a cross-cluster transition relative to a within-cluster transition \parencite{kahn2018network,kahn2018network,lynn2020abstract}.} To model this observed difference in reaction times and link them to the apparent differences shown in Figure \ref{fig:SR-TCM-model-simulations}, we apply principles of information theory. Specifically, we assume that response time for each stimulus is a function of the uncertainty in its surrounding context \parencite{fitts1964information}. Measures of information entropy have previously been used to explain RT differences between cluster transitions while traversing similar graph structures -- \ac{where higher entropy, which implies more uncertainty or more information available for a participant to process leads to higher response times \parencite{lynn2020abstract, lynn2020human,lynn2020humans}.} Formally, 

\begin{equation}
	\begin{aligned}
		RT(node) \cong Entropy(node) = \sum_{s' \in S} \hat{M}(s, s') * log(\hat{M}(s, s'))
	\end{aligned}
\end{equation}

where $M(s, s')$ is the context representation at node $s$. For SR, this expression evaluates to the expected future visits to state $s'$ from state $s$ whereas for TCM this expression evaluates to the extent to which $s'$ is activated as a result of $s$. 

As noted previously, a common indicator of participants having acquired the global structural knowledge is a slowdown in responses when the ongoing stimulus stream crosses a cluster (relative to transitions within a cluster) of the modular graph. Context representations can be used to model the cross cluster-transitions by computing a `surprisal' effect. For simulations, the surprisal effect is computed as the Jenson-Shannon distance between the context representations of two nodes. Formally, 

\begin{equation}
	\begin{aligned}
		RT(s \rightarrow s') \cong JS(s, s') = \sqrt[2]{\frac{D(M(s, .) || p) + D(M(s', .) || p)}{2}} \\
	\end{aligned}
\end{equation}

where $M(s, .)$ is the context representation of node vector $s$, $p$ is the point-wise mean of nodes $s$ and $s'$ and $D(M||p)$ is the Kullback-Leibler divergence between probability distributions $M$ and $p$. \ac{Jenson-Shannon distance thus scales with differences between the two context representations. Intuitively, since the representations of nodes 1 and 2 are similar (as shown in Figure \ref{fig:SR-TCM-model-simulations}), the Jenson-Shannon distance between these two nodes will be smaller than nodes 1 and 6. A direct measure or surprisal derived from the context matrix was also considered (See Figure \ref{fig:surprisal-sims} in the appendix for details)}

The formalization of observed response time differences due to surprisal (and node entropy) allows us to simulate expected reaction time distributions for novel walk types. Specifically, to understand the mechanisms behind acquiring the global modular graph pattern following a limited exposure \ac{(for example, infering the target store's organization via limited exploration over multiple visits)}, each model was simulated for random walk with lengths of 0, 3, 6, and 999. A random walk length of 0 translates to a completely random selection of one of the 15 nodes of the modular graph on each trial. Walk length of 3 and 6 translate to a random walk visiting 3 and 6 edges (4 and 7 nodes) respectively before being reset to a random node (similar to visiting the Target store in short bursts to purchase relevant items and checking out without visiting the entire store). Finally, a walk length of 999 translates to visiting 999 edges (with repetition) through their connections on the modular graph. \ac{Parameters of the simulations in Figure \ref{fig:SR-TCM-walklength-matrices} are determined through the same best-fitting RSA procedure described above.} 

\begin{figure}
	\centering
	\includegraphics[width = \textwidth]{chapter_notebooks/chapter_2/figures/walk_length_SR_TCM_matrices.png}
	\caption{\ac{Example} model predictions of context representations for SR and TCM models across different walk lengths for \ac{one specific set of parameters}. Both models seemingly predict that the modular structure of the original graph is increasingly recovered with longer walk lengths.}
	\label{fig:SR-TCM-walklength-matrices}
\end{figure}

The acquisition of the global structure can be modeled using surprisal as has been done in previous research \parencite{lynn2020abstract,lynn2020humans,lynn2020human}. To investigate the differences between models for various walk lengths and relate to measurable response time differences,  for a subset of parameters in the valid range of 0 to 1, each model was simulated to produce a context matrix. Jensen-Shannon distance was computed between each pairs of nodes and averaged over cross-cluster pair and within cluster pairs. Simulation results below show the transition Jensen-Shannon distances over 100 simulations of the model for each parameter combination. For SR, `param\_a' is the learning rate parameter $\alpha$ and `param\_b' is the discount parameter $\gamma$. For TCM, `param\_a' is the learning rate parameter $\alpha$ and `param\_b' is the context drift parameter $\rho$. 
\begin{figure}
	\centering
	\includegraphics[width = \textwidth]{chapter_notebooks/chapter_2/figures/SR_TCM_boundary_nonboundary_jsdist.png}
	\caption{\yk{Simulated} model predictions for differences in surprisal comparing across cluster transitions to within cluster transitions across walk lengths \yk{for a range of possible parameter values}. Both models predict that cross cluster surprisal effect will increase with walk length leading to an increased reaction time.}
	\label{fig:SR-TCM-walklength-transition-sjdist}
\end{figure}

In Figure \ref{fig:SR-TCM-walklength-transition-sjdist} \ac{The Y-axis represents the difference in surprisal between boundary nodes and non-boundary nodes as measured by Jensen-Shannon distance. The X-axis represents the parameter $\alpha$ and different hues represent parameters $\gamma$ for the SR model and $\rho$ for the TCM model. Top row are predictions for the SR model and bottom row for the TCM. Columns represent predictions for different walk lengths.}. 

The figure thus shows that both context models predict an increased surprisal \ac{in cross-cluster transitions relative to within-cluster transitions} as walk length through the modular graph gets longer. As walk length increases, context associated with each node increasingly represents neighboring nodes. Since neighbors of the boundary nodes \js{\ac{are largely within the cluster of that boundary node, representations of boundary nodes becomes increasingly similar to that of the non-boundary nodes within the same cluster, and thus increasingly different than boundary nodes in the neighbouring cluster. Thus, crossing a cluster (from a boundary node to another) leads to an increased surprisal of having encountered a node that is representationally dissimilar to the previous node. On the other hand, non-boundary nodes within the same cluster get increasingly closer in their representations with other non-boundary nodes in that cluster. Thus transitions between non-boundary nodes within a cluster does not increase surprisal.}}

The two context models, however, differ in their predictions in the role of a boundary node. \ac{The Y-axis of Figure \ref{fig:SR-TCM-walklength-boundary-nonboundary-entropydiff} shows the difference in entropy between boundary nodes over a range of parameters (X-axis and hue) for both SR and TCM models (rows) across the 4 walk lengths (columns).} SR predicts an increased entropy in its representation of the boundary nodes with walk length relative to the non-boundary nodes for some values of the $\alpha$ and $\gamma$ parameters. On the other hand the TCM does not predict such increased in boundary vs non-boundary entropy differences. 

\begin{figure}[ht]
	\centering
	\includegraphics[width = \textwidth]{chapter_notebooks/chapter_2/figures/SR_TCM_walklength_boundary_nonboundary_entropydiff.png}
	\caption{\yk{Simulated} model predictions of differences between the SR and the TCM after different walk lengths for \yk{a range of possible parameter values}. SR predicts that entropy of boundary nodes will scale with walk lengths whereas TCM does not.}
	\label{fig:SR-TCM-walklength-boundary-nonboundary-entropydiff}
\end{figure}


The predictive nature of SR (as modeled by the future discount, $\gamma$ parameter) allows for a representation of nodes in the neighboring cluster to impact entropy on the boundary node of the current cluster that leads to that neighboring cluster. This effect is unique on boundary nodes of a cluster as non-boundary nodes of the second cluster are closer to the immediate neighbor of the current cluster (i.e. the boundary node that serves as an entry point to the second cluster). Since TCM is associative (as opposed to predictive), only nodes that are `active' in representation impact the representation of the just experienced node thereby. This mechanism thus reduces the impact of the non-boundary nodes in neighboring cluster. Rescaled heatmap in figure \ref{fig:zoomed-in-SRTCM-boundary-entropy} presents this effect.

\begin{figure}[ht]
	\centering
	\includegraphics[width = \textwidth]{chapter_notebooks/chapter_2/figures/SR_vs_TCM_Matrices_zoomed.png}
	\caption{Rescaled SR and TCM matrices depict differences between context representations of the two models. Boundaries in SR incorporate more information than those in TCM.}
	\label{fig:zoomed-in-SRTCM-boundary-entropy}
\end{figure}

The SR-based predictive context representation in particular shows that boundary nodes carry more information than non-boundary nodes whereas the associative context representation does not produce this effect. \footnote{The activity in the lower third of both matrices is due to recency; while these are interesting patterns, and seem to indicate that SR can account for the recency effects in memory which was the primary motivation behind introduction of the TCM \parencite{gershman2012successor,howard2005temporal}. Investigating recency effects in this implicit statistical learning context is out of scope for this dissertation.}

Thus, both SR and TCM models would predict slow down in cross-cluster transitions relative to within cluster transitions, and that this slow down will increase with walk length. However, predictive context representations through SR are unique in predicting the scaled slow down at boundary nodes with random walk length, \textit{independent} \js{of where the boundary node has been visited from but as a result of the boundary nodes' inherent role in serving as a gateway between clusters}. While lack of a scaled slow down to boundary nodes does not invalidate the SR model (because some values of the parameters allows SR to not scale the slowed reactions with walk length), the presence of such a slow down provides evidence for predictive representations in such statistical learning tasks. The study presented next, thus tests this prediction. 

\section{Experiment 1: Testing Context Representations for implicit event boundaries}

\subsection{Methods}

\subsubsection*{Participants}
125 undergraduate students at the University of Massachusetts Amherst participated in this study for course credit. Data from 12 participants who did not complete the study was disregarded from further analyses. All study protocols were approved by the university institutional review board. \ac{Participant sample size was not pre-determined via a statistical procedure but was a rough equivalent of previous studies \parencite{kahn2018network,schapiro2013neural}}. \footnote{All inferences in this work are made in form of the probabilities of an effect as estimated via Bayesian analyses.}

\subsubsection*{Design and Procedure}

\ac{The general experimental procedure was similar to the one used by \textcite{kahn2018network}}. Participants were randomly assigned into one of four between-subject groups. All procedures for participants in all groups remained the same except for experimentally defined walk-lengths. Participants sat in an isolated room with an LCD computer screen operated by Windows 7. The experiment was designed using Psychopy \parencite{peirce2007psychopy}. 

As shown in Figure \ref{fig:exp1-design}, at the beginning of the study, participants were instructed to place their right hand on the computer keyboard such that their fingers aligned on the appropriate keys. On each trial, participants were presented with five grey boxes. One or two of the five grey boxes were highlighted using green borders. Participants were instructed to hit the combination of keys \yk{(simultaneously in cases where two keys were required to be hit)} corresponding to the highlighted boxes as fast as possible without making any errors. A trial did not end until participants hit the correct combination of keys. Participants were informed of their incorrect key presses and a trial where participants did not hit the correct combination of keys on the first try was marked as an inaccurate trial. The experiment lasted for 1400 trials, or 1 hour, whichever came first. To prevent fatigue, participants were given self-paced breaks after every 200 trials. Data from participants who did not complete all 1400 trials was discarded and not used for further analyses. At the end of the study, participants were debriefed about the research question of the study.

\begin{figure}
	\centering
	\includegraphics[width = \textwidth]{chapter_notebooks/chapter_2/figures/exp1_task_design.png}
	\caption{Task design for experiment 1. \textit{Left panel} modular graph used to generate random walks. \textit{Middle panel} Each node is randomly assigned to a combination of one or two highlighted boxes. \textit{Right panel} Participants place their hands on the keyboard as shown and are instructed to press keys that correspond to highlighted boxes.}
	\label{fig:exp1-design}
\end{figure}

Each of the 15 possible key combinations (where one or two of the grey boxes are highlighted) were randomly assigned to a node of the graph in \ref{fig:modular_graph}. Trial sequences were generated based on walk lengths. For all walk lengths, the first trial was selected at random. For walk lengths of 1399 (29 participants), the subsequent trials followed a random walk through the graph structure along the edges with edges connected. For walk lengths of 3 and 6, trials proceeded on a similar random walk for 3 (29 participants) and 6 (29 participants) edges respectively (thus visiting 4 and 7 nodes) before resetting to any of the fifteen nodes. Finally for random walk of length 0 (29 participants), each node of the 15 was picked with equal probability on each trial. 


\subsubsection*{Data Processing and Preliminary analyses}
All inaccurate trials (around 9.2\%) were removed from further analyses. Furthermore, accurate trials with response times beyond 3 standard deviations of the global response times (around 1.1\%) were removed from further analyses as well. In all, around 10\% of the total trials were discarded. 

\subsection{Results}
Table \ref{tab:exp1-rt-stats} shows the descriptive statistics \ac{(means, medians, and standard deviations)} of response time for each node and transition types \ac{aggregated over participants in each condition. The `Transition Type' column refers to the transition experienced immediately prior to a particular node where `within cluster' transition is the transition to a boundary or a non-boundary node from that same cluster and `cross cluster' transition is a transition to a boundary node from a boundary node of a neighbouring cluster. `Node Type' refers to the role of the current node in the graph of the left panel in Figure \ref{fig:exp1-design}.}

\begin{table}	
	\centering
	\caption{Response times descriptive statistics (in seconds) for experiment 1.}
	\label{tab:exp1-rt-stats}
	\begin{tabular}{lllrrr}
		\toprule
		 &  &  & \multicolumn{3}{r}{rt} \\
		 &  &  & mean & std & median \\
		transition type & node type & walk length &  &  &  \\
		\midrule
		\multirow[t]{4}{*}{cross cluster} & \multirow[t]{4}{*}{Boundary} & 0 & 0.954 & 0.565 & 0.786 \\
		 &  & 3 & 0.963 & 0.585 & 0.774 \\
		 &  & 6 & 0.973 & 0.594 & 0.785 \\
		 &  & 1399 & 0.990 & 0.596 & 0.802 \\
		\cline{1-6} \cline{2-6}
		\multirow[t]{8}{*}{within cluster} & \multirow[t]{4}{*}{Boundary} & 0 & 0.996 & 0.565 & 0.822 \\
		 &  & 3 & 0.980 & 0.598 & 0.787 \\
		 &  & 6 & 0.943 & 0.572 & 0.769 \\
		 &  & 1399 & 0.953 & 0.600 & 0.767 \\
		\cline{2-6}
		 & \multirow[t]{4}{*}{Non Boundary} & 0 & 0.963 & 0.565 & 0.790 \\
		 &  & 3 & 0.936 & 0.545 & 0.772 \\
		 &  & 6 & 0.985 & 0.616 & 0.790 \\
		 &  & 1399 & 0.932 & 0.592 & 0.747 \\
		\cline{1-6} \cline{2-6}
		\bottomrule
		\end{tabular}
	\end{table}
	

As expected, response times decreased with practice for all nodes in all conditions \mh{(Tables \ref{tab:first-last-blocks-3}, \ref{tab:first-last-blocks-6}, and \ref{tab:first-two-blocks-1399} for statistical comparisons)}. The median response times separated by transition type and node types are shown in Figures \ref{fig:rt-walklength-transitions} and \ref{fig:rt-walklength-nodes} respectively. \ac{Note that responses to Boundary node in Figure \ref{fig:rt-walklength-nodes} are a combination of within- and cross-cluster transitions.}

\begin{figure}
	\centering
	\includegraphics[width = \textwidth]{chapter_notebooks/chapter_2/figures/median_rts_transitiontype.png}
	\caption{Median response times \ac{at nodes following a transition} for each walk length separated by the type of transition.}
	\label{fig:rt-walklength-transitions}
\end{figure}


\begin{figure}
	\centering
	\includegraphics[width = \textwidth]{chapter_notebooks/chapter_2/figures/median_rts_nodetype.png}
	\caption{Median response times for each walk length separated by node types.}
	\label{fig:rt-walklength-nodes}
\end{figure}

Response time graphs show the overall pattern \ac{\textbf{expected from model simulations above}} -- cross cluster transitions are slower in longer walk lengths than within cluster transitions. Similarly, responses to boundary nodes are slower at longer walk lengths than responses to non-boundary nodes. 


\subsubsection*{Modeling}
The key question of interest is whether response times to boundary nodes slow down further \textit{after} accounting for slowdowns due to transitions. However, the effects of node type (comparing to non-boundary nodes) wll necessarily include effects of transition type since boundary nodes are accessed through both within cluster and across cluster transitions. Therefore, in order to isolate the effects of node type, response time differences between boundary and non-boundary nodes for within-cluster transitions were compared; thereby removing the effects of cross cluster transitions. Since transitions for walk length of 0 were random, this condition was also removed from further analyses. Similarly, all reset transitions in walk lengths of 3 and 6 \ac{(where the random walk ended and the subsequent node was picked between any of the 15 possible nodes)} which were \textit{not} a part of the random walk were discarded from further analyses \ac{to isolate the effect of the transition structure during a random walk (since nodes following resets do not follow the transition structure)}. 

Each block in the experiment consisted of 200 trials. Thus, by the end of the first block, participants in longer walk length conditions may have already experienced the graph structure sufficiently enough to acquire knowledge of the graph, thereby slowing down at the boundary nodes. While characterizing the entire 1400 trial learning curve \mh{\js{\ac{is ideal to measure the differences in reaction times of the critical transitions (non-boundary to non-boundary node compared with non-boundary to boundary node transitions), such linear model fit leads to the differences between reaction times at boundary and non-boundary nodes to map onto different parameters of the (see Appendix for a linear model of the entire learning curve) across different walk length making across walk length comparisons difficult}}}. Thus, to assess acquired patterns, the first two blocks of the data are compared using the following Bayesian model where standardized log response times were fit as a skewed normal distribution, separately for each walk length.

\begin{equation}
	\begin{aligned}
		node\ transition\ type : block \sim \mathcal{N}(0, 0.5) \\ 
		transition\ experience \sim \mathcal{N}(0, 0.2) \\  
		lag \sim \mathcal{N}(0, 0.2) \\ 
		\mu = node\ transition\ type : block + lag + transition\ experience \\
		\sigma \sim Exponential(1) \\ 
		skewness \sim \mathcal{N}(0, 3) \\ 
		log(RT) \sim Skew\mathcal{N}(mu, sigma, skewness)
	\end{aligned}
\end{equation} 

Where $node\ transition\ type$ is either non-boundary to non-boundary or non-boundary to boundary; block is 0 or 1, lag is the number of trials before which the current key combination was seen and transition experience is the number of times a particular transition leading into the current node was previously experienced. Figure \ref{fig:bayesmodel-firsttwoblocks} shows the Bayesian estimates of differences between walk lengths. \ac{The X-axis of the posterior histogram represents an estimate ofthe differences between response times at boundary nodes and non-boundary nodes in the second block (Block 1) after controlling for responses in the first block (Block 0) separeted by walk lengths of 3, 6, and 1399.}

\begin{figure}[h]
	\centering
	\includegraphics[width = 0.8\textwidth]{chapter_notebooks/chapter_2/figures/nb_b_diff.png}
	\caption{\ac{Estimated} differences in response times to boundary and non-boundary nodes when they are transitioned into from the same cluster (i.e. another non-boundary node). When accounting for the response times in the first block, as walk length increases, response times in the second block are increasingly slower to boundary nodes than non-boundary nodes.}
	\label{fig:bayesmodel-firsttwoblocks}
\end{figure}

In particular, participants in walk length of 1399 experienced the largest gains in response times of non-boundary to non-boundary transitions relative to those of non-boundary to boundary transitions from the first block to the second. As expected from the Successor Representation (SR) model, these gains were reduced for walk length of 6 and further so for walk length of 3 implying varying levels of structure acquisition depending on walk length. This pattern, which is uniquely expected in the SR model and not the TCM, thus provides support for predictive representations driving the formation of implicit event boundaries. 

\section{Discussion}

The primary aim of this work was to characterize the creation of implicitly operationalized event boundaries as a function of context representations. In particular, two models of context representations were contrasted: the associative TCM model and the predictive SR model. Both models express an increase in reaction time when crossing boundary nodes into a new cluster in the three-module graph structure (Figure \ref{fig:modular_graph}) as available context at boundary nodes across clusters drastically differs with each boundary node strongly representing events within its own cluster. However, the SR model \ac{expresses} the importance of boundary node as carrying additional information (measured by information theoretic entropy). The SR model predicts that as the `quality' of exposure (here operationalized by length of random walk) increases, the apparent importance of boundary nodes increases as well. 

To test this qualitative prediction of the SR (and thereby compare it with the TCM representation), a serial reaction time task was conducted with participants experiencing the modular graph at 4 different lengths of a random walk. As predicted by the SR, response times at boundary nodes slowed down the most for the longest random walk, and less so for shorter random walks. 

The experimental findings in this chapter thus provide support for maintaining a predictive representation of our environment \mh{and that associative, Hebbian mechanisms, are not enough to explain the observed data}. This error-driven predictive representation, which does not rely on explicit rewards, naturally leads to learning the statistical regularities in the environment and is thus crucial in informing our understanding of statistical learning and pattern acquisition. 

\mh{While SR is not unique in its expression of increased boundary information for a modular graph used in this work, other formulations (where increased boundary information is a result of erronous estimation of the transition probability) in prior work have been linked to closely follow the SR model \parencite{lynn2020abstract,lynn2020humans,lynn2020human}. Furthermore, this SR formulation allows us to use a neuro-psychologically plausible model in understanding pattern extraction and statistical learning \parencite{stachenfeld2017hippocampus,gershman2018successor,momennejad2017successor}. Finally, such predictive representations set stage for use of the SR (or SR-like) models over associative models to understand the broader work in event cognition and event boundaries \parencite{rouhani2020reward}.} 

In the current work parameters of the SR (or the TCM) model are not directly estimated as SR and TCM do not provide a direct measure of reaction times. While the assumption of reaction times scaling with increased available information (entropy) is logical, this assumption needs further testing. Future tests of such predictive representation should incorporate parameter estimation and hence also check the validity of the relationship between reaction time and information entropy. Similarly, a simpler model comparing two blocks was used to make inference in this task. More complex models (such as the exponential or the multi-rate state space models \parencite{savalia2024leap, smith2006interacting, mcdougle2015explicit}) should aim to characterize the entire learning curve to understand when participants start to acquire (and use) existing patterns. 

\mh{An SR based predictive representation, on its own is likely not sufficient to explain all patterns in the data. In some cases, participants may become explicitly aware of an existing structure, leading to a model-based reinforcement learning intervention on reaction times \parencite{momennejad2017successor} or constraints on working memory may require further considerations in how transition probabilities are learned \parencite{mcdougle2021modeling}. Future work should account for these possibilities in understanding the cognitive processes that underlie statistical learning.  Finally, findings presented in this work are limited to a single graph structure. Prior work has found that graphs of different topologies produce similar effects in cross-cluster slowdowns \parencite{karuza2019human} and future work should examine the modeling and experimental differences for a range of graph structures to assess whether findings in this work are dependent on the specific topological structure used.}  

\section{Conclusion}
The findings in this chapter provide evidence in favor of using predictive representations (as opposed to associative representations) to account for event boundaries that are operationalized implicitly. Findings in the current chapter do not distinguish between specific algorithms that lead to predictive representations; future work could contrast potential differences in these algorithms. It however remains unclear whether `boundary' nodes are truly so in context of event cognition -- the experiment presented in this chapter (and similar past literature) does not test whether stimuli at boundary nodes follow similar properties as stimuli at boundaries when events are operationalized through explicit context change. In the next chapter, I explore how such implicitly operationalized boundaries share properties with boundaries that are operationalized explicitly. 

\newpage
\chapter{Comparing Implicit Event Boundaries with Explicit Event Boundaries}
\label{chapter-3-implicit-explicit-event-boundaries}
\section{Introduction}
We receive a continuous stream of sensory information in our daily lives. In order to make sense of it, we often parse it into meaningful chunks for storage, retrieval and comprehension. For example, we may recall our drive to work as a series of discrete events; got into the car, got coffee, picked up a collegue, hit traffic on a particular street, parked, and walked over to the office. What aspects of the incoming stream help us organize continuous temporal information in such discrete chunks? 
dezfouli2014habits
Temporal chunking in cognitive psychology has been studied under several domains from event boundaries \cite{clewett2019transcending, zacks2007event, rouhani2020reward,rouhani2018dissociable,dubrow2013influence,baldwin2008segmenting}, language learning, \cite{romberg2010statistical,knowlton1992intact}, categorization \cite{unger2022ready,gabay2015incidental}, and motor sequencing \cite{bera2021motor, tremblay2010movement, savalia2016unified,ostlund2009evidence}. Chunking a repeated sequence of experiences is crucial to abstracting patterns in the environment and formation of habits for quick and efficient interactions with the environment \cite{dezfouli2012habits, smith2016habit,dolan2013goals, dezfouli2014habits, gershman2010learning, botvinick2012hierarchical}. 

Models of temporal event segmentation suggest that the points which lead to temporal segmentation seem to be unique in their properties in both segmenting the continuous stream of information and integration of information across the temporal event. These `event boundaries' are, for example, shown to be remembered better \cite{swallow2009event,rouhani2018dissociable,rouhani2018dissociable, zacks2020event, radvansky2017event, heusser2018perceptual}, serve as points of retrieval \cite{michelmann2023evidence} and replay to promote long term memory \cite{hahamy2023human, sols2017event} and easy parsing, help integrate memory across time \cite{clewett2019transcending}, and separates across boundary events while collapsing within boundary events \cite{clewett2019transcending, lositsky2016neural,ezzyat2014similarity, brunec2018boundaries}. 

In most prior studies, event boundaries have been studied using explicit context shifts. For example, when stream of stimuli are surrounded by colored border, event boundaries are operationalized by first showing the stimuli surrounded by a color and abruptly changing that color\cite{heusser2018perceptual}. In another study, event boundaries were operationalized via explicit context changes by changing the associated stimulus\cite{ezzyat2014similarity}. A pair of images were presented on each trial; one image of the pair, the 'scene' image remained constant for a short sequence of trials whereas the other ('object' or 'face') changed on each trial. Participants were asked to make judgments about the object/face image \cite{ezzyat2014similarity}. Previously, context changes had been operationalized as either perceptual or semantic shift in ongoing set of events by having participants watch clips \cite{swallow2009event}. In more recent work, context change has been operationalized as changes in ongoing reward contingencies associated with each stimulus \cite{rouhani2020reward}. 

Consistent findings across most studies in explicitly operationalized event boundaries show that event boundaries are often remembered better \cite{swallow2009event, radvansky2017event, heusser2018perceptual,clewett2019transcending, rouhani2020reward,ezzyat2014similarity}, and events across boundaries appear to be perceptually farther whereas events within boundaries appear to be perceptually closer \cite{clewett2019transcending,ezzyat2014similarity,brunec2018boundaries,lositsky2016neural}. In recent work, however, it has been shown that event boundaries can also be formed \textit{without} explicit changes in context. After being exposed to a stream of stimuli such that the ordering is controlled by a modular graph shown in figure \ref{fig:modular_graph}, participants seem to recognize across cluster transitions as `natural breaks' more often than within cluster transitions \cite{schapiro2013neural}. In recent work, this finding has been linked to statistical learning of temporal graph structures \cite{karuza2022value,karuza2019human,kahn2018network,kahn2018network,lynn2020abstract,lynn2020human,lynn2020humans} and measured by slowed reaction times across clusters than within clusters. However, past studies where boundaries are operationalized implicitly do not assess the memory representations of these boundaries using the same tests used in explicitly operationalized boundary paradigms. 

In this chapter, I present two tests on implicitly operationalized boundaries to assess whether they elicit the same behavioral properties as the explicitly operationalized boundaries. In particular, I use the paradigm and graph structure previously used in Schapiro et al. \cite{schapiro2013neural} to test whether participants recall boundary items better (or worse) than non-boundary items. I then use a two module graph structure in Figure \ref{fig:two_module_graph} to test whether items across the two clusters appear farther than items within a cluster (similar to findings in explicitly operationalized boundary paradigms). 

\section{Modeling Boundary Effects}
Event segmentation theory suggests that the segmentation of the continuous sensory experience occurs automatically\cite{swallow2009event}. Past work on event boundaries (when operationalized explicitly) provide for a role of 

\section{Context Modeling and Predictions}
Chapter \ref{chapter-2-walk-lengths-modulate-statistical-learning} show that context models can be used to estimate representations of implicitly operationalized event boundaries. Particularly, predictive representations such as the SR provide a natural representation of event boundaries which form bottlenecks in transitioning between clusters in modular graphs in figure \ref{fig:modular_graph}. I propose that the same context-representation framework can be used to model memory differences. For the purposes of simulations, I use the REM (Retrieving Effectively from Memory) model in recognition memory\cite{shiffrin1997model}.

\chapter{Category Learning Through Temporal Abstraction}
\label{chapter-4-category-learning-through-temporal-abstraction}
\section{(Partly polished) Introduction}

We naturally categorize items we encounter daily for ease of storage, processing, and decision-making. For example we know instinctively that regardless of shape and form, all lamps form a 'lamp' category based on its function. Several factors determine how we categorize items. Depending on the complexity of rules that determine categories, some categorizations are easier than others \cite{shepard1961learning, nosofsky1994comparing}. Category variability can modulate how often exemplars are classified into that categories \cite{cohen2001category}.

The order of presentation items in category learning tasks has been shown to be an important factor in how category diagnostic features are learned. In particular, when items are presented in a blocked categorical design, participants seem to learn the similarities between the same category items. On the other hand, when items are presented as an interleaved design, participants seem to focus more on learning the features that differentiate the underlying categories \cite{carvalho2017sequence}. As a result of order-dependent differing focus on category diagnostic features, interleaved presentations seem to benefit general category learning. In most prior category-learning tasks assessing order of presentation effects, participants are explicitly asked to learn the underlying categories. There appear to be clear differences when participants focus on learning categories based on how exemplars of these categories are presented \cite{kornell2008learning, kornell2010spacing, whitehead2021transfer, vlach2008spacing, carvalho2014putting, carvalho2017sequence}. In this article, we investigate the effects of order of presentation when category learning is implicit. 

One primary focus on category learning through order of presentation is comparing blocked or interleaved exemplar presentations. For example, \cite{kornell2008learning} showed participants paintings made by two different painters. The order of presentation during exposure was modulated to either be blocked (paintings of one artist shown together followed by the second artist) or interleaved (paintings made by both artists were mixed). When presented with new paintings, and asked which of the two studied artists made them, participants who were exposed to the interleaved format were found to be more accurate at guessing the creator. Category learning also improved for interleaved presentation compared to blocked presentation when tested on items where relevant category features were visually occluded \cite{whitehead2021transfer}. When three-year-old children are tested on the generalization of category-specific features, they appear to benefit from the spaced study of exemplars as compared to a blocked \cite{vlach2008spacing}. By modulating the similarity of presented items, interleaved presentation was found to be better than blocked presentation design on generalization performance particularly when learned exemplars were more visual \cite{kornell2008learning, carvalho2014putting}. 

Interleaved presentation has been theorized to improve in category induction because of context-based variability during encoding \cite{glenberg1979component}. Particularly, for each presented item, an observer will store both the item-specific features along with the context in which the item is encoded. During interleaved presentation, a category diagnostic feature gets encoded under different contexts. Thus, that diagnostic feature will be recalled when tested on novel category items within that context. 

Two theories have been proposed to explain this discriminability-based advantage of interleaving. According to the attention attenuation account, when categories are blocked, participants may think that they have learned the relevant category features after viewing a few items and stop paying attention to additional exemplars of the \cite{kornell2010spacing}. On the other hand, according to the discrimination account, the interleaved presentation allows participants to directly compare the differences between exemplars of different categories that are presented close to each other thereby highlighting these differences \cite{kornell2008learning}. In a direct test \cite{wahlheim2011spacing} found that when participants were shown pairs of exemplars, each belonging to a different category, the interleaving benefit was magnified compared to when they were presented as single items. The authors posit that showing pairs of exemplars would enable participants to carefully study and infer distinctions between category features and hence improve categorization performance .Furthermore, the authors find evidence against the attention attenuation theory by observing that classification performance did not differ as a function of the position in which the exemplar was presented in a stream.

This benefit of interleaved presentation is shown to be modulated by the `level' at which categorization occurs. For example, when \cite{mack2015dynamics} modulated exposure time to individual exemplars along with order of exposure, they found that interleaved presentation was no longer beneficial under short exposure conditions particularly when participants were asked to make a more abstract, `super-ordinate level categorization. On the other hand, when exemplars were presented in a blocked format, a lower, `basic' level categorization was hindered. Thus, category knowledge through order of presentation can be modulated by the level of categorization participants are asked to produce.

It is clear that order of presentation of categories matters during explicit category learning. \textbf{However, the effect of such order of presentation has not been investigated when category learning is implicit}. Indeed recent work shows that participants do acquire category knowledge that when presented implicitly instead of being explicitly asked to learn categories. \cite{unger2022ready} found that assessed on category knowledge, participants appeared to learn category structures without being explicitly instructed to do so. This category knowledge was modulated by the strength of association of the category diagnostic features. \cite{unger2023without} later found that when presented with implicit feature-based categories during a cover task, participants were sensitive towards category diagnostic knowledge. 

More evidence for incidental category learning comes from auditory cognition. \cite{gabay2015incidental} found that participants were sensitive to audio categories learned implicitly as measured by increased reaction times when audio-category-to-response mapping was altered. Incidental category knowledge is further modulated by the sampling category distributions from which exemplars are drawn. \cite{roark2018task} show that probabilistic sampling of exemplars leads to weaker category learning compared to deterministic sampling. Incidental category learning can be further enhanced by task-relevant and disrupted by task-irrelevant feature-to-category mappings \cite{roark2022representational}. Incidental learning may also be disadvantaged compared to supervised intentional learning when categories are non-linearly separable \cite{love2002comparing}.

Thus, while implicit category learning appears to be consistent and dependent on several aspects of the underlying categories, unlike explicit category learning, it is unclear whether implicit category learning enjoys the same advantage when category exemplars are presented in an interleaved vs. a blocked design. Most implicit categorization tasks involve manipulation of features as opposed to manipulation of the temporal order of exposure. In this article, we explore the effects of temporal co-occurrences to manipulate


\paragraph{Other stuff to incorporate:} 

Unsupervised category learning \cite{billman1996unsupervised}, participants could infer rules based on correlating features without being explicitly asked to categorize during exposure. 

Category-related items were recognized better when presented close to each other then when category-unrelated items were \cite{medin1994presentation}. In \cite{medin1987family}, people unsupervised sorted stimuli by a single dimension, ignoring the family resemblance structure. 

SUSTAIN \cite{love2004sustain} seems to explain all these unsupervised category learning phenomena. ALCOVE \cite{kruschke2020alcove} provides for error based diagnostic feature attention learning in GCM \cite{nosofsky2011generalized, nosofsky1986attention}.


\subsection{Participants}


\subsection{Materials}



\chapter{General Discussion and Conclusion}
We experience a stream of continuous information daily. For ease of reference and recall, we break this stream down and store it in meaningful chunks. Event segmentation is a cognitive process that provides mechanisms to break down this temporal stream of information and integrate it across multiple chunks. Prior work has focused on understanding the event cognition process. In the spirit of decomposing cognitive processes into representations and operations on those representations \parencite{cowell2019roadmap}, in this dissertation I explore how this process can be assessed through shared representations of temporal structure in memory and operations on these representations. 

In the second chapter, I investigate how representations of ongoing context can produce behavioral patterns in response times. To that end, I contrast two models of context -- the Successor Representation (SR) as a predictive model that maintains context as an expectation of the future and follows error-driven learning with the Temporal Context Model (TCM) as an associative model that maintains representation as current activity and follows Hebbian learning. I show that the two models produce qualitatively different predictions when exposed to varying amounts of limited information about the environment. I then test these predictions in a serial reaction time task and show that data are more consistent with predictions from the SR model.

\mh{Findings in this chapter provide a theoretical framework with which event boundaries and statistical learning could be studied. Prior work has suggested that slow-down at boundary nodes may be due to a closely related algorithmic process where errors in realization of the temporal structure \parencite{lynn2020abstract}. In addition to the differences in walk lengths explored in this chapter, the predictive SR framework further allows to make testable predictions about information available at different stages of the learning process. SR further provides an implementation level mechanisms via dopaminergic projections to the Hippocampus may allow for error-driven learning of statistical regularities\parencite{stachenfeld2017hippocampus,gershman2018successor}. Future work should aim to directly investigate the differences in predictions between the two algorithmic accounts (erroneous learning vs predictive representation).}

\ac{One major assumption in chapter \ref{chapter-2-walk-lengths-modulate-statistical-learning} warrants some discussion. I assume that higher information contained in a node's representation (derived via SR or TCM) translates to slower reaction times. While there is prior support for slower responses relating higher information in similar tasks \parencite{lynn2020human}, this measure of response time is indirect. It is possible that the reaction time increases at a given node are a result of the source node of the incoming transition rather than the node itself -- previous research have found that as node degree increases, reaction times increase \parencite{lynn2020human}. Nevertheless, in the comparisons tested for experiment 1, source nodes are always non-boundary nodes. Thus, an observed effect must be due to the type of node at which the responses are made. In this work, I made a direct comparison between two models of context representations, the TCM and SR. Future work should also consider other models of context representations and assess how these models of associative memory differ in such statistical learning tasks \parencite{estes1955statistical, mensink1988model, murdock1997context}. Finally, in the experiments used in this chapter, no distinction has been made between what aspect of the participant's experience is associated with a node in the graph in Figure \ref{fig:modular_graph}. Future work should distinguish whether the effects of nodes predicted by models are due to effects on, for example, the visual stimulus associated with those nodes, the motor response associated with that node (which is in turn associated with the visual stimulus), or a combination of both.}

Prior research and findings in event cognition are often limited to tasks where event boundaries are defined by explicit context changes (e.g. change of scene in a movie). Recently, event boundaries have also been shown to be formed without such explicit context changes but through an underlying temporal structure. However, these implicit event boundary tasks are not typically tested using the same tests used for explicit event boundaries. In Chapter \ref{chapter-3-implicit-explicit-event-boundaries}, across two experiments I assess whether implicitly operationalized boundaries share behavioral properties with explicit event boundaries by (1) Testing whether they are remembered better than non-boundaries and (2) Testing whether events across boundaries are perceived farther than those within. Results from the experiment recognition memory experiment provide support for shared representations between implicit and explicit boundaries. Results from the distance judgment experiment do not provide sufficient evidence for such shared representations. However, combined results from both experiments provide support for SR-based context representations for implicit event boundaries.

\mh{Chapter \ref{chapter-3-implicit-explicit-event-boundaries} provides an important indication on shared representations between explicitly operationalized event boundaries and implicitly operationalized event boundaries and whether implicit boundaries are indeed event boundaries. While they have been labeled as boundaries in prior work \parencite*{schapiro2013neural}, work in this chapter provides a more direct test of whether implicit boundaries share behavioral properties of explicit boundaries. The SR modeling framework further provides a representational account for implicit boundaries. Future work should test whether explicit boundaries can be similarly represented through a representational framework such as the SR. While some modeling work has used the associative Context Maintenance and Retrieval Model \parencite{rouhani2020reward} to understand explicit event boundaries, findings in Chapter \ref{chapter-2-walk-lengths-modulate-statistical-learning} suggest that a predictive model may be more appropriate.}

\mh{Chapter \ref{chapter-3-implicit-explicit-event-boundaries} is limited to two tests that are used in explicit boundary paradigms. Future work should aim at testing whether implicit boundaries share other properties of explicit boundaries such as to serve as access points in memory \parencite{michelmann2023evidence}, points in replay \parencite{sols2017event, hahamy2023human}, and points of memory integration \parencite{griffiths2020event}. Systematic assessment of shared properties between implicit and explicit event boundaries will provide further insights into general structure of temporal cognition and pattern recognition.}

Finally in Chapter \ref{chapter-4-category-learning-through-temporal-abstraction}, I show that predictive representation of temporal events (such as SR) can be further used to understand the cognitive process of category learning. Prior findings have suggested that there is a differential benefit to presenting different category exemplars in an interleaved fashion vs blocked fashion in category learning. The SR model provides a representational account for this temporal order of presentation effects by modulating attention towards category diagnostic features. While the precise nature of this modulation operation further depends on the specific visual features that define categories, implicit visual category learning tasks in this chapter show, as predicted by SR, that the temporal order of presentation of category exemplars indeed matters in how categories are learned.

\mh{Work in Chapter \ref{chapter-4-category-learning-through-temporal-abstraction} thus provides a framework for understanding higher order cognition and representation. While categorization and category learning in cognitive psychology is well studied, current work provides a deeper algorithmic insight into how we learn categories in a natural, unsupervised manner. The experimental paradigms used can be further adapted to address other higher order cognitive functions of learning as well. For example, a common problem often addressed through Hierarchical Reinforcement Learning \parencite{botvinick2012hierarchical}, navigation through space can be thought of as achieving a set of sub-goals where all experiences within a sub-goal can be thought of a category.}

\ac{While the framing of studies presented in this dissertation is via event boundaries or statistical learning, the representational framework does not necessarily distinguish between these two processes. Chapters \ref{chapter-2-walk-lengths-modulate-statistical-learning} and \ref{chapter-3-implicit-explicit-event-boundaries} provide an important connection between these two fields of event boundaries and statistical learning (See also \cite{perruchet2006implicit}). Event boundaries, which are implicitly operationalized, are learned over time and provide an important marker to separate statistical regularities in the experienced environment. In turn, extracting of statistical patterns provides a natural break between two statistical regularities that are sufficiently different from each other thereby leading to event boundaries. The representational framework used here provides a way to algorithmically combine the findings in these two fields of cognition.} 

\js{This dissertation assesses shared representations between implicit and explicit event boundaries. One could also argue that boundaries that naturally became explicit were also originally learned implicitly. For a child growing up, the items in the kitchen are no different than the items in the living room. Nevertheless, over time, the child begins to identify the patterns by interacting (or watching others interact) with these items in different contexts. The shared context between kitchen items slowly consolidates into coherent, abstract knowledge that a kitchen is where one eats. Similarly, the shared context within the living room items slowly leads to the abstract knowledge that the living room is where one hangs out (albeit supported by instructions from parents). This way, talking to a friend in the living room becomes a separate event from talking to the same friend in the kitchen. Parent instruction notwithstanding, this formation of boundaries, which are explicit boundaries for adults, was originally acquired by extracting regularities in the environment through statistical learning.}

\js{Statistical learning, a cognitive process through which we acquire patterns in our environment, is widely applicable across multiple domains of cognitive psychology (See \cite{schapiro2015statistical} for a brief review). Our brain supports learning of such regularities automatically and implicitly. Learning the regularities in our environment allows us to develop useful heuristics to make quick decisions when needed. For example, when we visit a new grocery store, we know to generally expect all medications arranged in a specific section separate from a section of food items. This abstract knowledge of how grocery stores work allows us to perform quicker visual searches through aisle headers depending on what we are looking to shop for. The key question this dissertation seeks to answer is what algorithms our brain may implement in order to acquire abstract knowledge of these patterns.}

\ac{Overall, this dissertation shows that representing a temporal sequence of events such that each event represents a prediction of what might happen next, provides \textit{one} reasonable explanation of how we may learn statistical regularities from our environment. This dissertation further shows that such a predictive representation then provides a common framework for investigation into various aspects of human cognition such as forming event boundaries, statistical learning, or categorization. Typically, theoretical work in cognitive psychology is aimed at understanding individual cognitive processes that allow us to function. This representational framework opens a way for cognitive scientists to understand the broader role of such implicit, unsupervised error-driven learning. In future work more cognitive processes such as learning and decision-making in complex environments, the role of cognitive flexibility in behavior, the impact of the environmental experience on emotional and other affective internal states, and others can be distilled into such a common predictive representational account, thereby allowing for a more coherent understanding of human brain function.} 

\section{Broader Implications}



\mh{The spatiotemporal hypothesis in event cognition suggests that items that are closer together in both space and time share representations in the hippocampus \parencite{turk2019hippocampus}. The representational framework proposed in this dissertation suggests that this finding can be extended to other aspects of human psychology as well. For example, the cross-race effect studied in social psychology and eyewitness identification \parencite{young2012perception, wilson2013cross} where recognition for same race faces than different race faces could be studied through a lens of shared representation in how often exposure through development is higher with other faces of the same race.} 

\mh{Errors that lead to learning of of representations such as SR are often linked to dopaminergic signaling in the midbrain \parencite{gershman2018successor}. Future work should test whether increased availability of dopamine during adolescence which has been shown to enhance memory may also allow for a better pattern recognition during that period of development \parencite{cohen2022reward}. Impacts on pattern recognition due to varying dopamine availability over development could have further implications for instruction and in teaching complex concepts across different ages. Similarly, future studies should test whether depleted dopamine levels due to Parkinson's or Huntington's would on the other hand deter pattern recognition and statistical learning.}


\mh{Finally, the focus of studies cognitive psychology (and for this dissertation) has often been limited to participant population in the global north. It is clear that even in meaningless stimuli, the nature of exposure impacts low level cognition of memory and perceptual categorization. Future research must therefore test the generalizability of this representational framework in heterogenous participant population. Recent research in cultural effects on cognition suggests differing processes underlying `lower' level cognition of numbers and time \parencite{pitt2018metaphorical}. Such molding of cognitive processes could also be therefore viewed through the lens of differing representations with consistent operations on those representations as function of different cultural exposures.}

It has been argued that boundaries separating events temporally and spatially share representations. This dissertation further advocates for shared (algorithmic) representations across different processes in cognitive psychology. I show that a common predictive representation framework can be useful in understanding and relating implicit statistical and motor learning (chapter \ref{chapter-2-walk-lengths-modulate-statistical-learning}), event cognition and memory (chapter \ref{chapter-3-implicit-explicit-event-boundaries}), and categorization (chapter \ref{chapter-4-category-learning-through-temporal-abstraction}). Using a common representation can provide an important algorithmic constraint in understanding different cognitive processes. Future work in cognitive psychology can therefore focus on testing operations on these common representations that may account for observed behavior.


\appendix
\chapter{Chapter 2}

\section{Comparing first and the last blocks}

Similar to comparisons done in the main text, the response times between the boundary and non boundary nodes for the first and last blocks were compared when transitions leading into those nodes were within cluster (i.e. from another non-boundary nodes).

\begin{figure}[H]
    \centering
    \includegraphics[width = 0.75\textwidth]{chapter_notebooks/chapter_2/figures/nb_b_diff_block60.png}
    \caption{Posterior estimates of comparisons the slowed down reaction times for boundary nodes relative to non boundary nodes. Reaction times slowed down more with larger walk lengths.}
    \label{fig:bayesmodel-firstlastblocks}
\end{figure}

Figure \ref{fig:bayesmodel-firstlastblocks} provides further evidence in support of the SR model. Reaction times decrease across the board. The decrease is lower for boundary nodes than non boundary nodes and this difference increases with walk length providing support for the SR model. 

\section{Fitting the entire learning curve}
The entire learning curve was fit using a linear model to estimate differences in decreased response times between boundary and non boundary nodes over time. 

\begin{figure}[H]
    \centering
    \includegraphics[width = \textwidth]{chapter_notebooks/chapter_2/figures/allblocks_lrmodel_trial_ppt_lag.png}
    \caption{Intercept (\textit{Left panel}) and Slope (\textit{Right panel}) differences of the linear model fit to all trials.}
    \label{fig:alltrial-lrmodel}
\end{figure}

As shown in figure \ref{fig:alltrial-lrmodel}, boundary and non boundary intercepts differed across walk lengths. This difference is likely due to randomly assigned response keys. The key metric of differences is the reduction in response times over time as measured by the slopes. As expected (from the SR model), walk lengths of 3 and 1399 lead to slower decrease in response times for boundary nodes than non boundary nodes (when transitioned to from another within cluster (non boundary) node). 

Interestingly, it appears that these slope differences are not meaningfully different between walk lengths of 6 and 1399. However, the findings reported in the main text (Figure \ref{fig:bayesmodel-firsttwoblocks}), it is likely that these differences were apparent earlier in the learning process for the walk length of 1399 than for walk length of 6. Conflicting results with the main text likely implies a need for a more complex model (such as a dual rate model \cite{mcdougle2015explicit, smith2006interacting, savalia2022leap}) which allows for a quick initial decrease in reaction times followed by a slow asymptote. Future modeling work should investigate the impact of walk lengths on different aspects of the learning process.

\section{Model Statistics}

\subsection*{Stats for comparisons between the RTs for the first two blocks}

\begin{table}[H]
    \centering
    \begin{tabular}{lrrrr}
        \toprule
         & mean & sd & hdi 2.5\% & hdi 97.5\% \\
        \midrule
        beta block[0.0] & -0.066 & 0.176 & -0.409 & 0.269 \\
        beta block[1.0] & -0.080 & 0.177 & -0.417 & 0.271 \\
        beta lag & 0.039 & 0.007 & 0.026 & 0.053 \\
        beta transition exp & -0.018 & 0.004 & -0.025 & -0.011 \\
        beta trial ntr[boundary, 0.0] & -0.555 & 0.024 & -0.602 & -0.508 \\
        beta trial ntr[boundary, 1.0] & -0.629 & 0.028 & -0.685 & -0.574 \\
        beta trial ntr[nonboundary, 0.0] & -0.596 & 0.023 & -0.639 & -0.548 \\
        beta trial ntr[nonboundary, 1.0] & -0.650 & 0.028 & -0.702 & -0.596 \\
        \bottomrule
        \end{tabular}        
        \caption{Posterior parameter statistics for model fit to walk length 3 data comparing the first two blocks}
        \label{tab:first-two-blocks-3}
\end{table}

\begin{table}[H]
    \centering
    \begin{tabular}{lrrrr}
        \toprule
         & mean & sd & hdi 2.5\% & hdi 97.5\% \\
        \midrule
        beta block[0.0] & 0.143 & 0.173 & -0.191 & 0.482 \\
        beta block[1.0] & 0.124 & 0.181 & -0.225 & 0.475 \\
        beta lag & 0.043 & 0.008 & 0.028 & 0.058 \\
        beta transition exp & -0.031 & 0.004 & -0.038 & -0.023 \\
        beta trial ntr[boundary, 0.0] & 0.252 & 0.033 & 0.190 & 0.319 \\
        beta trial ntr[boundary, 1.0] & 0.184 & 0.037 & 0.111 & 0.257 \\
        beta trial ntr[nonboundary, 0.0] & 0.316 & 0.033 & 0.252 & 0.383 \\
        beta trial ntr[nonboundary, 1.0] & 0.241 & 0.037 & 0.165 & 0.310 \\
        \bottomrule
    \end{tabular}
    \caption{Posterior parameter statistics for model fit to walk length 6 data comparing the first two blocks}
    \label{tab:first-two-blocks-6}
\end{table}

\begin{table}[H]
    \centering
    \begin{tabular}{lrrrr}
        \toprule
         & mean & sd & hdi 2.5\% & hdi 97.5\% \\
        \midrule
        beta block[0.0] & -0.043 & 0.172 & -0.378 & 0.294 \\
        beta block[1.0] & -0.076 & 0.177 & -0.427 & 0.267 \\
        beta lag & 0.066 & 0.007 & 0.053 & 0.080 \\
        beta transition exp & -0.010 & 0.003 & -0.016 & -0.005 \\
        beta trial ntr[nonboundary, 0.0] & -0.476 & 0.023 & -0.522 & -0.431 \\
        beta trial ntr[nonboundary, 1.0] & -0.635 & 0.027 & -0.688 & -0.581 \\
        beta trial ntr[boundary, 0.0] & -0.499 & 0.024 & -0.546 & -0.455 \\
        beta trial ntr[boundary, 1.0] & -0.602 & 0.027 & -0.655 & -0.548 \\
        \bottomrule
        \end{tabular}   
        \caption{Posterior parameter statistics for model fit to walk length 1399 data comparing the first two blocks}
        \label{tab:first-two-blocks-1399}    
\end{table}

\subsection*{Stats for comparisons between the RTs for the first and last blocks}

\begin{table}[H]
    \centering
    \begin{tabular}{lrrrr}
        \toprule
         & mean & sd & hdi 2.5\% & hdi 97.5\% \\
        \midrule
        beta block[0.0] & -0.070 & 0.173 & -0.392 & 0.274 \\
        beta block[6.0] & -0.100 & 0.172 & -0.431 & 0.236 \\
        beta lag & 0.029 & 0.006 & 0.017 & 0.041 \\
        beta transition exp & -0.007 & 0.002 & -0.010 & -0.003 \\
        beta trial ntr[boundary, 0.0] & -0.561 & 0.020 & -0.601 & -0.523 \\
        beta trial ntr[boundary, 6.0] & -0.704 & 0.037 & -0.778 & -0.634 \\
        beta trial ntr[nonboundary, 0.0] & -0.604 & 0.020 & -0.642 & -0.563 \\
        beta trial ntr[nonboundary, 6.0] & -0.730 & 0.039 & -0.808 & -0.656 \\
        \bottomrule
    \end{tabular}        
    \caption{Posterior parameter statistics for model fit to walk length 3 data comparing the first and the last blocks}
    \label{tab:first-last-blocks-3}    
\end{table}

\begin{table}[H]
    \centering
    \begin{tabular}{lrrrr}
        \toprule
         & mean & sd & hdi 2.5\% & hdi 97.5\% \\
        \midrule
        beta block[0.0] & -0.054 & 0.175 & -0.406 & 0.279 \\
        beta block[6.0] & -0.099 & 0.178 & -0.460 & 0.228 \\
        beta lag & 0.038 & 0.006 & 0.026 & 0.050 \\
        beta transition exp & -0.006 & 0.002 & -0.010 & -0.003 \\
        beta trial ntr[boundary, 0.0] & -0.538 & 0.019 & -0.574 & -0.501 \\
        beta trial ntr[boundary, 6.0] & -0.734 & 0.038 & -0.809 & -0.660 \\
        beta trial ntr[nonboundary, 0.0] & -0.528 & 0.020 & -0.568 & -0.489 \\
        beta trial ntr[nonboundary, 6.0] & -0.700 & 0.038 & -0.777 & -0.629 \\
        \bottomrule
    \end{tabular}
    \caption{Posterior parameter statistics for model fit to walk length 6 data comparing the first and the last blocks}
    \label{tab:first-last-blocks-6}    
    
\end{table}

\begin{table}[H]
    \centering
    \begin{tabular}{lrrrr}
        \toprule
         & mean & sd & hdi 2.5\% & hdi 97.5\% \\
        \midrule
        beta block[0.0] & -0.053 & 0.178 & -0.404 & 0.306 \\
        beta block[6.0] & -0.122 & 0.176 & -0.458 & 0.230 \\
        beta lag & 0.060 & 0.006 & 0.049 & 0.072 \\
        beta transition exp & -0.003 & 0.001 & -0.006 & -0.001 \\
        beta trial ntr[nonboundary, 0.0] & -0.515 & 0.019 & -0.553 & -0.478 \\
        beta trial ntr[nonboundary, 6.0] & -0.845 & 0.034 & -0.910 & -0.778 \\
        beta trial ntr[boundary, 0.0] & -0.538 & 0.020 & -0.576 & -0.499 \\
        beta trial ntr[boundary, 6.0] & -0.825 & 0.034 & -0.891 & -0.762 \\
        \bottomrule
    \end{tabular}        
    \caption{Posterior parameter statistics for model fit to walk length 1399 data comparing the first and the last blocks}
    \label{tab:first-last-blocks-1399}    
\end{table}


\subsection*{Parameter statistics for linear model including all trials}

\begin{table}[H]
    \centering
    \begin{tabular}{lrrrr}
        \toprule
         & mean & sd & hdi 2.5\% & hdi 97.5\% \\
        \midrule
        alpha ntr[3, boundary] & 0.466 & 0.138 & 0.215 & 0.760 \\
        alpha ntr[3, nonboundary] & 0.376 & 0.138 & 0.106 & 0.650 \\
        alpha ntr[6, boundary] & 0.440 & 0.138 & 0.166 & 0.699 \\
        alpha ntr[6, nonboundary] & 0.539 & 0.138 & 0.261 & 0.789 \\
        alpha ntr[1400, boundary] & 0.489 & 0.134 & 0.281 & 0.826 \\
        alpha ntr[1400, nonboundary] & 0.484 & 0.133 & 0.263 & 0.802 \\
        beta ntr[3, boundary] & -0.000 & 0.000 & -0.001 & -0.000 \\
        beta ntr[3, nonboundary] & -0.000 & 0.000 & -0.001 & -0.000 \\
        beta ntr[6, boundary] & -0.000 & 0.000 & -0.001 & -0.000 \\
        beta ntr[6, nonboundary] & -0.001 & 0.000 & -0.001 & -0.000 \\
        beta ntr[1400, boundary] & -0.001 & 0.000 & -0.001 & -0.001 \\
        beta ntr[1400, nonboundary] & -0.001 & 0.000 & -0.001 & -0.001 \\
        \bottomrule
    \end{tabular}        
    \caption{Parameter statistics for linear model fitting all trials. Parameter `alpha' is the intercept, and parameter `beta' is the slope of the linear model.}
    \label{tab:allblocks-trial-ppt-lag}         
\end{table}

\chapter{Chapter 3}

\begin{table}[H]
    \centering
    \caption{Bayesian SDT Model results for boundary nodes from experiment 3a. }
    \label{tab:exp3-bayesmodel-boundary-sdt}
    \begin{tabular}{lrrrr}
        \toprule
         & mean & sd & hdi 3\% & hdi 97\% \\
        \midrule
        accuracy exposure & -0.805 & 0.716 & -2.088 & 0.591 \\
        true old|structured & 4.088 & 0.332 & 3.472 & 4.711 \\
        true old|unstructured & 4.200 & 0.308 & 3.611 & 4.783 \\
        true old|condition sigma & 5.228 & 2.233 & 1.934 & 9.368 \\
        \bottomrule
        \end{tabular}        
\end{table}

\begin{table}[H]
    \centering
    \caption{Bayesian SDT Model results for non boundary nodes from experiment 3a. }
    \label{tab:exp3-bayesmodel-nonboundary-sdt}
    \begin{tabular}{lrrrr}
        \toprule
         & mean & sd & hdi 3\% & hdi 97\% \\
        \midrule
        accuracy exposure & -1.347 & 0.715 & -2.665 & -0.020 \\
        true old|structured & 3.885 & 0.269 & 3.398 & 4.406 \\
        true old|unstructured & 4.593 & 0.299 & 4.017 & 5.122 \\
        true old|condition sigma & 5.156 & 2.002 & 2.010 & 8.873 \\
        \bottomrule
        \end{tabular}
        
\end{table}

\begin{table}[H]
    \centering
    \caption{Drift diffusion model parameters for experiment 3a.}
    \label{tab:exp3-ddm-params}
    \begin{tabular}{lrrrrrrrrr}
        \toprule
        parameter, condition, block & mean & sd & hdi 3\% & hdi 97\% \\
        \midrule
        a structured, 0 & 1.259 & 0.022 & 1.220 & 1.300 \\
        a structured, 1 & 1.144 & 0.021 & 1.104 & 1.183 \\
        a structured, 2 & 1.121 & 0.021 & 1.081 & 1.159 \\
        a unstructured, 0 & 1.306 & 0.023 & 1.263 & 1.349 \\
        a unstructured, 1 & 1.237 & 0.025 & 1.193 & 1.284 \\
        a unstructured, 2 & 1.149 & 0.023 & 1.107 & 1.191 \\
        v boundary, structured, 0 & 0.385 & 0.092 & 0.202 & 0.546 \\
        v boundary, structured, 1 & 0.476 & 0.096 & 0.294 & 0.653 \\
        v boundary, structured, 2 & 0.688 & 0.097 & 0.506 & 0.873 \\
        v boundary, unstructured, 0 & 0.554 & 0.108 & 0.357 & 0.753 \\
        v boundary, unstructured, 1 & 0.761 & 0.101 & 0.565 & 0.944 \\
        v boundary, unstructured, 2 & 0.830 & 0.106 & 0.637 & 1.029 \\
        v new, structured, 0 & -0.733 & 0.062 & -0.848 & -0.618 \\
        v new, structured, 1 & -1.124 & 0.077 & -1.267 & -0.980 \\
        v new, structured, 2 & -1.011 & 0.075 & -1.159 & -0.879 \\
        v new, unstructured, 0 & -0.872 & 0.062 & -0.986 & -0.754 \\
        v new, unstructured, 1 & -1.414 & 0.081 & -1.560 & -1.259 \\
        v new, unstructured, 2 & -1.267 & 0.079 & -1.408 & -1.112 \\
        v nonboundary, structured, 0 & 0.236 & 0.069 & 0.102 & 0.362 \\
        v nonboundary, structured, 1 & 0.468 & 0.081 & 0.326 & 0.631 \\
        v nonboundary, structured, 2 & 0.570 & 0.084 & 0.411 & 0.730 \\
        v nonboundary, unstructured, 0 & 0.619 & 0.079 & 0.471 & 0.768 \\
        v nonboundary, unstructured, 1 & 0.830 & 0.092 & 0.663 & 1.005 \\
        v nonboundary, unstructured, 2 & 0.927 & 0.097 & 0.749 & 1.114 \\
        z 0 & 0.499 & 0.010 & 0.481 & 0.517 \\
        z 1 & 0.505 & 0.010 & 0.485 & 0.523 \\
        z 2 & 0.489 & 0.010 & 0.470 & 0.507 \\
        \bottomrule
        \end{tabular}
        
\end{table}

\begin{table}[H]
    \centering
    \caption{Bayesian model results for Experiment 3b.}
    \label{lab:exp3-bayesmodel-stats}
    \begin{tabular}{lrrrr}
        \toprule
        True Distance, Condition & mean & sd & hdi 3\% & hdi 97\% \\
        \midrule
        1, structured & 0.222 & 0.191 & -0.155 & 0.562 \\
        1, unstructured & 0.096 & 0.224 & -0.338 & 0.496 \\
        2, structured & -0.026 & 0.205 & -0.401 & 0.358 \\
        2, unstructured & 0.113 & 0.234 & -0.329 & 0.543 \\
        3, structured & -0.320 & 0.280 & -0.857 & 0.184 \\
        3, unstructured & -0.505 & 0.337 & -1.137 & 0.106 \\
        \bottomrule
        \end{tabular}
        
\end{table}

\chapter{Chapter 4}
\section{Model Statistics}
\begin{table}[H]
    \centering
    \caption{Bayesian model statistics for experiment 4a}
    \label{tab:exp4-bayes-model-results}
    \begin{tabular}{lrrrr}
        \toprule
        Num Features, Condition & mean & sd & hdi 3\% & hdi 97\% \\
        \midrule
        1.0, structured & 0.324 & 0.308 & -0.240 & 0.927 \\
        1.0, unstructured & 0.236 & 0.304 & -0.300 & 0.832 \\
        2.0, structured & 0.661 & 0.309 & 0.077 & 1.253 \\
        2.0, unstructured & 0.031 & 0.304 & -0.508 & 0.621 \\
        3.0, structured & 0.563 & 0.311 & -0.004 & 1.152 \\
        3.0, unstructured & 0.189 & 0.302 & -0.350 & 0.753 \\
        4.0, structured & 0.564 & 0.306 & -0.017 & 1.138 \\
        4.0, unstructured & 0.126 & 0.303 & -0.444 & 0.684 \\
        \bottomrule
        \end{tabular}
        
\end{table}

\begin{table}[H]
    \centering
    \caption{Bayesian model statistics for experiment 4b.}
    \label{tab:exp5-bayes-model-results}
    \begin{tabular}{lrrrr}
        \toprule
         Num Features, Category Determinant & mean & sd & hdi 3\% & hdi 97\% \\
        \midrule
        1.0, A & 1.005 & 0.372 & 0.303 & 1.693 \\
        1.0, B & -1.529 & 0.438 & -2.361 & -0.713 \\
        2.0, A & 1.338 & 0.378 & 0.620 & 2.036 \\
        2.0, B & -1.803 & 0.443 & -2.638 & -0.977 \\
        3.0, A & 1.299 & 0.378 & 0.551 & 1.987 \\
        3.0, B & -1.709 & 0.439 & -2.508 & -0.848 \\
        4.0, A & 1.687 & 0.385 & 0.977 & 2.435 \\
        4.0, B & -1.706 & 0.442 & -2.596 & -0.948 \\
        \bottomrule
        \end{tabular}
        
\end{table}

\section{Feature Differences}
Experiment 3b provides an indication that categorization (regardless of if it is dependent on temporal consistency) depends on the importance of each visual feature of the stimuli. At test, participants categorized the previously seen stimulus by matching its feature to the two on-screen options. On each test trial, the on-screen options differ from each other in one to four features such that for the `category' option all the category diagnostic features (those that remained temporally consistent during exposure) remained the same as the test stimulus. For the `non category' option, the features that were not category diagnostic (those that frequently changed during exposure) remained the same as the test stimulus. Thus, in order to categorize based on category diagnostic features, participants must \textit{ignore} the different valued non-category diagnostic features in the category option. Furthermore, participants must use the different category diagnostic feature in the non-category option to decide to not pick that option. 

A logistic regression model was fit where each feature was coded as whether it was identical in the non-category option (relative to the test stimulus). If a feature changed in the non category option, (hence was identical in the category option), it was coded as '1' otherwise it was coded as '0'. Fitting this model thus provided a measure of the weights a participant placed on that feature to select the category option. Figure 

\begin{figure}[H]
    \centering
    \includegraphics[width = \textwidth]{chapter_notebooks/chapter_4/figures/feat_importances.png}
    \caption{Relative weights placed by participants to categorize test stimuli. Values above 0 indicate features that are category diagnostic (i.e. remained consistent during exposure) are chosen more often to categorize whereas values below 0 indicate features that are category non-diagnostic (i.e. changed frequently) are used to categorize.}
    \label{fig:feat-importances}
\end{figure}

As seen in figure \ref{fig:feat-importances}, some features are grouped together for being temporally consistent whereas some features are grouped together for being temporally inconsistent. Interestingly, this is true even in the unstructured case. Regardless of the temporal exposure, participants will group aliens together if they have the same head color or nose shape. On average, features that were chosen to be category diagnostic for Category A participants in experiment 3b (head, eyes, foot, and nose), were inherently grouped together for sharing feature values. Whereas features that were chosen to be category diagnostic for Category B participants in experiment 3b (torso, arms, antenna, bellybutton) were inherently grouped together if the did \textit{not} share feature values. While this analysis does not address covariances among features, it provides a possible explanation for patterns in Experiment 3b -- visual features carry different weights in categorization. Based on their weights, attention may be drawn to them either for being temporally consistent or for being temporally inconsistent. Future modeling will aim to assess whether attention weights modulate 

\interlinepenalty=10000  % prevent split bibliography entries

% \bibliography{references}
% \bibliographystyle{umassthesis}

	
\printbibliography

\end{document}

%% 
%% Copyright (C) 2019 by Daniel A. Weiss <daniel.weiss.led at gmail.com>
%% 
%% This work may be distributed and/or modified under the
%% conditions of the LaTeX Project Public License (LPPL), either
%% version 1.3c of this license or (at your option) any later
%% version.  The latest version of this license is in the file:
%% 
%% http://www.latex-project.org/lppl.txt
%% 
%% Users may freely modify these files without permission, as long as the
%% copyright line and this statement are maintained intact.
%% 
%% This work is not endorsed by, affiliated with, or probably even known
%% by, the American Psychological Association.
%% 
%% This work is "maintained" (as per LPPL maintenance status) by
%% Daniel A. Weiss.
%% 
%% This work consists of the file  apa7.dtx
%% and the derived files           apa7.ins,
%%                                 apa7.cls,
%%                                 apa7.pdf,
%%                                 README,
%%                                 APA7american.txt,
%%                                 APA7british.txt,
%%                                 APA7dutch.txt,
%%                                 APA7english.txt,
%%                                 APA7german.txt,
%%                                 APA7ngerman.txt,
%%                                 APA7greek.txt,
%%                                 APA7czech.txt,
%%                                 APA7turkish.txt,
%%                                 APA7endfloat.cfg,
%%                                 Figure1.pdf,
%%                                 shortsample.tex,
%%                                 longsample.tex, and
%%                                 bibliography.bib.
%% 
%%
%%
%% This is file `./samples/shortsample.tex',
%% generated with the docstrip utility.
%%
%% The original source files were:
%%
%% apa7.dtx  (with options: `shortsample')
%% ----------------------------------------------------------------------
%% 
%% apa7 - A LaTeX class for formatting documents in compliance with the
%% American Psychological Association's Publication Manual, 7th edition
%% 
%% Copyright (C) 2019 by Daniel A. Weiss <daniel.weiss.led at gmail.com>
%% 
%% This work may be distributed and/or modified under the
%% conditions of the LaTeX Project Public License (LPPL), either
%% version 1.3c of this license or (at your option) any later
%% version.  The latest version of this license is in the file:
%% 
%% http://www.latex-project.org/lppl.txt
%% 
%% Users may freely modify these files without permission, as long as the
%% copyright line and this statement are maintained intact.
%% 
%% This work is not endorsed by, affiliated with, or probably even known
%% by, the American Psychological Association.
%% 
%% ----------------------------------------------------------------------