We experience a constant stream of sensory information from the moment we are born. Our brain parses this information and slowly learns to extract meaning from it. From recognizing the mother's scent as a survival instinct to formulating complex plans to defeat a board game opponent, our brain extracts meaning from our surroundings, considers prior experience and the current state of the world, and makes decisions to interact accordingly. My key question in this dissertation is: How do we learn to extract meaning from our surroundings?

Extracting meaningful information from our experience is beneficial to our functioning and survival. When approached by a cheetah in a forest, we do not wait to evaluate the exact number of spots on its skin before deciding to run. Such abstractions and formation of heuristics allow us to process naturally complex surroundings in quick time and act accordingly.

When extracting meaning from our surroundings, we often (need to) ignore minute details and abstract out towards a coherent thought of our surroundings. The extreme example above aside, we observe such abstractions in almost all day-to-day activities. Imagine someone asking you about the events during the day before reading this manuscript. Perhaps you are reading it on your office computer and you recall a sequence of events starting from waking up, to getting ready, driving or taking public transit to work, getting your morning coffee and breakfast in with an email check before turning your attention to this manuscript. Each event described above combines several sub-events that are abstracted away in such a verbal recall and description. For example, getting ready involves several steps from brushing your teeth, showering, and wearing your work outfits. Each sub-event can be further thought of as an abstraction from sub-sub-events -- brushing your teeth is a combination of putting toothpaste on the brush head, the physical act of brushing, followed by rinsing. While we perform each act continuously in time, our recall (and by extension, representation in memory) of these past events is discrete, segmented, and abstracted. The meaningfully separatable chunks of a continuous stream of events help in storage in (and retrieval from) memory.

There is often agreement on what it means for a chunk and for boundaries defining such chunks to be `meaningful'. In the example above, it is reasonable to argue that brushing teeth, showering, and putting on clothes are three distinct activities. Furthermore, even when the true transitions between these activities are continuous and seamless to a  independent, naive observer, the boundaries between these events are perceptually meaningful. What aspect of the environment dictates this agreement about the points at which we segment events and what is special about the properties of the events between those points that lead to distinct representations in memory?

One could argue that these events that occur at different points in time also occur at different spatial locations, thus providing different contexts and hence separate representations in our brain to be considered distinct. Transitioning from one (temporal) event to another can thus be akin to transitioning from one room of the house to another. However, it is almost impossible to decouple temporal events from spatial events assuming a causality from spatial segmentation to temporal segmentation for any spatially experienced distinct event is also a temporally experienced distinct event. Instead, arguing that temporally distinct events lead to a spatially distinct representation can provide a more encompassing explanation of distinct representations for events in distinct spatial \emph{and} temporal contexts. One could similarly argue that the formation of ``meaningful'' chunks is through perceptual differences between the events we experience. While lower-level perceptual experiences are indeed often different for different events, the mapping of perceptual differences onto segmented events is arbitrary and not a \emph{sufficient} condition. For example, eating an apple is recalled as eating an apple regardless of whether it has a green leaf added to its top. Perceptual distinctions are not enough to determine whether events are represented distinctly. What then is the key mechanism that leads to events being segmented? 

In this dissertation, I argue that the primary reason and mechanism through which we segment events is based on temporal contingencies of various sub-events that encompass an event. Specifically, we recall being in the kitchen as different from being in the bedroom because we have a coherent set of experiences in the kitchen that are distinct from a coherent set of experiences in the bedroom. For an infant forming knowledge of the world, a kitchen while perceptually distinct from a bedroom, is not meaningfully different. With experience and observation of the functions within these spatially (and perceptually) distinct locations, the child slowly develops distinct representations of the two rooms.

This dissertation focuses on understanding temporal contingencies' role in event segmentation, and by extension, general pattern extraction. I argue that even \emph{without} any spatial or perceptual information that may aid us in separating events in memory, we can use temporal coherence to experience separate events and abstract information to aid higher-level cognition. Specifically, I investigate the parsing of a continuous stream of information into discrete chunks in three ways:



\begin{itemize}
	\item
	The possible algorithmic representations that naturally lead to such segmentation and the impact of environmental properties in aiding this abstraction.
	\item
	The properties of the temporal boundaries when such temporal abstraction occurs naturally and implicitly.
	\item
	The role of temporal events separated by underlying transition structures in forming higher-order abstractions such as categories.
\end{itemize}

\section{Scope of this dissertation}\label{scope-of-this-dissertation}

The human brain is a complex machine -- millions of neurons act as computational units and combine in specific ways to form a functioning human being. These neurons come together to implement several levels of function from lower-level automatic perception to higher-level planning and conscious thought. This dissertation does \textbf{not} focus on these implementational-level mechanisms of cognition. Rather, it focuses on \emph{algorithmic} computations that neurons may, collectively, implement that lead to us acquiring patterns in the environment around us.

Analyses in this dissertation use several models of cognition in investigating the role of implicit statistical learning. In most cases, the focus of this dissertation is \textbf{not} to evaluate the validity of these models. Indeed, most models of cognition are wrong; but are useful and I use several such models to evaluate specific aspects of how we acquire patterns. Similarly, this dissertation proposes modifications to the previously known model based on context representations. These modifications are solely to derive predictions and provide possible explanations of findings from these models and are not rigorously tested. Future work should aim to test these model updates and therefore provide more holistic explanation of the cognitive processes explored in this manuscript.

Finally, the data collected and used in this dissertation is much more rich than presented. In order to limit the scope to the specific questions of interest, only a subset of analyses involving simpler (often linear) models is presented. Future work will incorporate more complex models on this data to extract fine grained information of the representation driven cognitive processes.

\section{Format of this dissertation}\label{format-of-this-dissertation}

In the rest of this dissertation, I present three lines of studies investigating the role of implicit temporal boundaries in cognition. In Chapter 2, I present an algorithmic representational framework that naturally leads to a representation of separable events without the need to rely on explicit properties of the experienced events. I also show that this predictive framework allows for a distinct representation of event boundaries as special events and contrast it with an associative representation. In Chapter 3, I present work comparing the properties of these event boundaries which are operationalized implicitly (i.e. through no perceptually special information) with event boundaries as they have been studied in prior literature which are operationalized explicitly. Finally, in Chapter 4, I present work investigating the role of these implicitly operationalized event boundaries in categorization to serve as a gateway for understanding higher-order cognition in the context of temporal segmentation and pattern acquisition.
